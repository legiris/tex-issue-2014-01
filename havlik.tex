\newpage\nuluj
\pagestyle{empty}

\logo

\nadpis{Vilnius Eastern Partnership Summit: Milestone in EU-Russia Relations -- not just for Ukraine}

{\azb abcdef}

\autor{Peter Havlik\pozn{The Vienna Institute for International Economic Studies (wiiw), Rahlgasse 3, A-1060 Vienna, Austria. E-mail: havlik@wiiw.ac.at.
The author wishes to thank Vasily Astrov and Vladimir Gligorov for useful comments on an earlier draft. The views expressed in this note and any remaining errors are the responsibility of the present author.}}
\linka{3ex}

\cast{Abstract}
The Vilnius Eastern Partnership Summit on 28--29th November 2013 represents a milestone in EU relations not just with respect to the six Eastern Partnership countries (EaP Armenia, Azerbaijan, Belarus, Georgia, Moldova and particularly Ukraine), but also with the EU?s ?strategic partner? Russia. The turbulence and numerous speculations regarding expectations about the signature of the EU-Ukraine Association Agreement (comprising a Deep and Comprehensive Free Trade Agreement ? AA/DCFTA), as well as progress in initialling similar future agreements with Georgia and Moldova, have been escalating before the summit. The association agreements would bring EaP signatory countries closer to the EU not really closer to EU membership, but closer to the application of various EU norms and standards (takeover of the ?acquis communautaire?) and ? significantly ? out of the Russian orbit, for the beginning at least symbolically. The last minute postponement of the EU-Ukraine AA/DCFTA signature announced by Ukraine's government just one week before the summit represents a serious setback for the EU. Though the EU has no ?Plan B? and was stunned after Ukraine?s announcement, life will continue after the summit and new initiatives will have to be started. What are the relevant issues and challenges and what is at stake? This note attempts to evaluate the consequences (economic and otherwise) of alternate decisions following the Vilnius Eastern Partnership Summit, reviews some of the disputed arguments and discusses selected relevant economic issues.\\

\cast{Keywords}
Vilnius Eastern Partnership Summit, European Union, Ukraine, EU-Ukraine Association Agreement REMOVE ``
\linka{4ex}

\podnadpis{Introduction}
The Vilnius Eastern Partnership Summit on 28-29 November 2013 represents a milestone in EU relations not just with respect to the six Eastern Partnership countries (EaP: Armenia, Azerbaijan, Belarus, Georgia, Moldova and particularly Ukraine), but also with the EU?s ?strategic partner?, Russia. The turbulence and numerous speculations regarding expectations about the signature of the EU-Ukraine Association Agreement (comprising a Deep and Comprehensive Free Trade Agreement ? AA/DCFTA), as well as progress in initialling similar future agreements with Georgia and Moldova, escalated before the summit. The association agreements would bring EaP signatory countries closer to the EU: not really closer to EU membership, but closer to the application of various EU norms and standards (takeover of the ?acquis communautaire?) and ? significantly ? out of the Russian orbit, at least symbolically at first. The postponement of the EU-Ukraine AA/DCFTA signature ? Ukraine?s government halted the related preparations just one week before the summit ? represents a serious setback for the EU, while Russia has gained another strategic point, at least for a while.\pozn{Ukraine?s government proposed the establishment of a tripartite commission with the EU and Russia in order to jointly discuss trade and economic issues ? see www.gazeta.ru, 21 November, 2013. The interruption of the AA/DCFTA process was presented by Ukraine?s Prime Minister Mykola Azarov as a ?tactical decision? driven solely by economic reasoning.} Though the EU has no ?Plan B? and EU High Representative Catherine Ashton expressed her disappointment immediately after Ukraine?s announcement, life will continue after the summit and new initiatives will have to be started. 

What are the relevant issues and challenges and what is at stake? This note briefly discusses the positions of the key individual parties (the European Union, Ukraine and Russia), presents details on foreign trade and tariff data  and attempts to evaluate the consequences (economic and otherwise) of alternate decisions following the Vilnius Eastern Partnership Summit. It also reviews some of the disputed arguments and discusses selected relevant economic issues.

\podnadpis{Tug-of-war over Ukraine}
In its present form, the conclusion and implementation of an AA/DCFTA between Ukraine and the EU has been presented by both the EU and Russia as incompatible with the participation of EaP countries in the Russian-led Customs Union (BRK-CU: the other members being Belarus and Kazakhstan) and especially with Ukraine joining the envisaged ?deeper and wider? post-Soviet integration project in the framework of the Eurasian Union and the Single Economic Space (SES).\pozn{The Eurasian Union (EurAz) currently includes, apart from Russia, Belarus and Kazakhstan, Kyrgyzstan and Tajikistan. The future Eurasian Union and SES envisages a common market entailing ?four freedoms? modelled on the EU experience.} Until compromise solutions regarding tariff regimes have been negotiated, the two directions for integration ? either with the EU or participation in the BRK-CU/SES ? are indeed incompatible. Russia?s ?success? in luring Armenia into the BRK-CU instead of opting for an AA/DCFTA with the EU, as announced on 3 September, 2013, was initially interpreted by some in the EU as incompatible with Armenia?s prospective conclusion of an AA/DCFTA. Later on, European Commissioner for Enlargement and Neighbourhood Policy �tefan F�le attempted to de-escalate tensions and tried to dismiss such fears by stating in October at a conference in Kyiv that the AA/DCFTA should ?not be seen as a threat but as an opportunity, a contribution to creating an area of free trade between Lisbon and Vladivostok?. Furthermore, he explained that the European Commission is ?working on the issue of legal incompatibility between the Association Agreement and Customs Union?, while requiring once again that Ukraine shows ?determined action? and delivers ?tangible progress on all European Union benchmarks?.\pozn{See F�le (2013a, 2013b). The latter requirement was spelled out by Mr F�le in a speech before the Ukrainian parliament in Kyiv on the same day (F�le, 2013c).} Ukraine, for its part, would opt for signing the AA/DCFTA agreement (a corresponding decision was already adopted by Ukraine?s government in September 2013) while, ideally, desiring to ?cherry pick? and maintain and develop good relations with both the Russian-led Customs Union and the EU. One of the EU?s key demands ? to cease the application of ?selective justice? and in particular the release of former Prime Minister Yulia Tymoshenko from prison ? will obviously not be fulfilled, at least not before the Vilnius Summit.\pozn{EU foreign affairs ministers reiterated conditions for signing the AA/DCFTA agreement at their meeting on 18 November, 2013, in Brussels as follows: ?Determined action and tangible progress is needed in three areas: the compliance of the parliamentary elections with international standards, addressing the issue of selective justice and preventing its recurrence, and the implementation of the reforms jointly agreed in the Association Agenda? (http://www.euractiv.com/specialreport-ukraine-way-reform/analysts-slam-germany-ukraine-po-news-531768).}

\podnadpis{Russia?s bullying pays off}
To deal first with Russia, there has been some history of this country?s use of economic sanctions in order to retaliate for perceived unwelcome political developments in the ?near abroad?. Russian sanctions ranged from import bans on Georgian wine and mineral water in 2004-06 after the ?Rose Revolution? in Georgia, the interruption of gas deliveries to Ukraine and Belarus related to disputes over pricing and access to pipelines, restricting the import of wine and spirits from Moldova, imposing import restrictions on dairy products and chocolate from Ukraine, on dairy product imports from Lithuania, etc.\pozn{It must be added, for the sake of completeness, that Russia also employed trade sanctions ? with varying justification ? with respect to imports of US poultry, Polish pork, Dutch flowers, etc.} The latest sore point in Russian external relations with potentially severe economic consequences has been Russia?s concentrated efforts in bullying Ukraine (as well as Georgia and Moldova) related to the envisaged AA/DCFTA signature at the forthcoming Eastern Partnership Summit in Vilnius.\pozn{See Moldova?s Foreign and European Integration Minister (who is also chief AA negotiator) Natalia Gherman at Euractiv.com, published on 30 October, 2013, and the interview with newly elected Georgian President Giorgi Margvelashvili in Kommersant Vlast?, No. 41, October 2013, respectively.} The frequency and intensity of Russia?s rather crude attempts to prevent Ukraine from signing the AA/DCFTA prior to the Vilnius Summit and to ?explain the adverse consequences of the signature?, together with simultaneous efforts to ?lure? Ukraine into joining the Russian-led Customs Union with Belarus and Kazakhstan, escalated before the Vilnius Summit.\pozn{There was even a ?secret? (although leaked) strategy for preventing Ukraine from signing the AA/DCFTA agreement published in August by the Ukrainian paper Zerkalo Nedely ? see http://gazeta.zn.ua/internal/o-komplekse-mer-po-vovlecheniyu-ukrainy-v-evraziyskiy-integracionnyy-process-\_.html.} Repeatedly, Sergey Glazyev, one of President Putin?s economic advisors, lectured Ukraine on the alleged adverse consequences of signing the ?discriminative? AA/DCFTA agreement while simultaneously threatening Russian sanctions. Indeed, Russian border controls on Ukrainian exports were briefly introduced (on a ?trial? basis, but still violating the CIS FTA agreements where Ukraine participates) while simultaneously praising the economic benefits of Ukraine joining the Customs Union. Similar warnings were spelled out by Russian Ambassador to the EU Vladimir Chizhov and reiterated, albeit in a more polite form, by Russian First Deputy Prime Minister Igor Shuvalov, who is in charge of EurAz economic relations in Russia.\pozn{See http://www.euractiv.com/europes-east/top-envoy-russia-offer-ukraine-e-news-530890 and http://www.euractiv.com/europes-east/russia-reiterates-warnings-ukrai-news-530671.} Last but not least, Russian Prime Minister Dmitry Medvedev warned his Ukrainian colleague Mykola Azarov that after signing the AA/DCFTA Ukraine will have ?zero chance? of full-scale CU membership, while Russian Foreign Affairs Minister Sergey Lavrov mentioned the possibility of tightened border controls between the two countries.\pozn{See report from the meeting of the two prime ministers in Kaluga on 15 October, 2013 (www.gazeta.ru/business/2013/10/15). For Lavrov?s speech, see http://www.gazeta.ru/politics/2013/10/28\_a\_5727929.shtml.} The latest serious and immediate threat was expressed by Mr Medvedev at the beginning of November in connection with Ukraine?s payment arrears for Russian gas deliveries (amounting to nearly USD 900 million as of August 2013). Prime Minister Medvedev required prompt debt repayment, rejected new Russian credit and required a pre-payment for additional gas deliveries (envisaged by the existing contract with Gazprom) while suggesting that, if necessary, Ukraine should ask the EU for financial assistance instead.\pozn{See www.gazeta.ru, 4 November, 2013.} Meanwhile, Ukraine is also being squeezed by the IMF, which is urging the government to reduce budgetary expenditures and raise domestic gas tariffs, as well as to implement a number of other unpopular reforms before resuming new financing.\pozn{See IMF Mission Statement to Ukraine, Press Release No.\;13/419, 31 October, 2013.}  

Russia?s bullying attempts to pressure its neighbours to ?integrate? with Russia instead of with the EU was seen as counterproductive not only by many Ukrainians and most outside observers, but even by a number of commentators in Russia.\pozn{See, for instance, http://www.vedomosti.ru/opinion/print/2013/10/29/18070451, K. Sonin and Financial Times, 4 November, 2013, p.\;9.} On the other hand, a negative view regarding the consequences of an AA/DCFTA signature is shared by Ukraine?s communists, who claim ? probably correctly ? that the country has no prospects of joining the EU in the next 20-30 years and that the implementation of EU regulations would be too costly, while EU integration is allegedly supported by just 40\% of Ukrainians.\pozn{See http://www.euractiv.com/europes-east/ukrainian-communists-expose-myth-news-531359. It must be added that the high costs and the rationality of the request to take over the ?acquis? and other provisions of the DCFTA without prospects for EU accession is criticised by other observers as well (Dreyer, 2012). The earlier experience of CEE NMS suggests that ?acquis takeover? is indeed costly and problematic (Havlik, 2003).} Last but not least, there have been tensions among current BRK-CU members as neither Belarus nor Kazakhstan ? the two other members of BRK-CU ? are particularly happy with current Russian dealings related to the CU stance. For example, at the recent BRK-CU summit in Minsk (end-October 2013), Kazakh President Nursultan Nazarbayev complained about the ?excessive politicisation? of the CU Commission?s decisions pursued by Russian representatives who are ?not independent enough? from the government (a situation which contradicts CU Commission statutes). Furthermore, Belarus President Alexander Lukashenko complained at the same summit about increased bureaucratic obstacles in BRK-CU customs procedures and delays in other integration steps.\pozn{See http://www.vedomosti.ru/politics/print/2013/10/25/17942981.} On the sidelines, Mr Nazarbayev also suggested inviting Turkey to join the BRK-CU.\pozn{Ibid. Note that Turkey has been in a customs union with the EU since the mid-1990s.}


\podnadpis{EU?s failed Eastern Partnership}
Following a number of resolute previous ?either/or? statements regarding the direction of integration by various EU representatives, European Commissioner for Enlargement and Neighbourhood Policy �tefan F�le attempted to de-escalate the situation, declaring that this issue ?is not a choice between Moscow and Brussels? and promised Ukraine a speedier AA/DCFTA implementation after the Vilnius summit. Mr F�le also declared that the European Commission is ?working on overcoming the issues of legal compatibility between the AA and CU? in order to ?prevent new walls in Europe?,\pozn{See Mr F�le?s speech at the conference in Yalta, Ukraine, on 20 September, 2013.} and sharply rebuked claims regarding the adverse effects of an AA/DCFTA.\pozn{See the above quoted speeches by Mr F�le at the international conference in Kyiv and before the Ukrainian Parliament on 11 October, 2013, at http://europa.eu/rapid/press-release\_SPEECH-13-808\_en.htm and 13-810 (ibid).} On the same day, Mr F�le announced a ?post-Vilnius agenda? for Ukraine which would include financial assistance to support the implementation of the Association Agreement amounting to EUR 186 million and move ahead with macro-financial assistance of EUR 610 million, ?once the conditions are in place? (ibid). The current EU stance with respect to both Eastern Partnership countries and the ?partnership for modernisation? with Russia, as well as the lack of a corresponding longer-term strategy, have long been criticised by numerous observers and experts.\pozn{See, for example, Wallace (2009), Grant (2011), Emerson (2011a) and recent statements by German Bundestag MP Karl-Gerog Wellmann and former EU Enlargement Commissioner Gunter Verheugen during a panel discussion in Berlin on 18 November, 2013, (http://www.euractiv.com/specialreport-ukraine-way-reform/analysts-slam-germany-ukraine-po-news-531768). Similar views have recently been expressed also by French diplomats (http://www.euractiv.com/europes-east/france-supports-association-agre-news-531726as), as well as by Youngs and Pishchikova (2013) and Wisniewski (2013).}

\podnadpis{Economic integration effects}
Available studies dealing with the (economic) effects of alternative integration agreements provide widely conflicting results, depending on methods, assumptions and data sources. One of the common findings of these studies is that (economic) effects on Russia (or the EU for that matter) are asymmetric: they are rather small compared to the effects on smaller prospective integration partners such as Ukraine, Armenia, Georgia or Moldova ? owing to the sheer size of the Russian/EU economy (see, among others, Astrov et al., 2012; EDB, 2012, 2013; Vinokurov and Libman, 2012; Dabrowski and Taran, 2012; Dreyer, 2012; Movchan and Shportyuk, 2012; EBRD, 2012). Another common finding of most integration studies is that (mostly short-term) tariff reduction effects are relatively small compared to the effects from the abolishment of non-tariff barriers and the expected medium- and long-term efficiency gains from capital inflows and related restructuring. This applies not only to post-Soviet integration or the AA/DCFTA but, for example, to the Transatlantic Trade and Investment Partnership (TTIP) under negotiation between the EU and US as well (see Francois et al., 2013).

Regarding the effects of alternative integration scenarios, there is a plethora of different evaluation approaches, applying various methodologies, assumptions and data sets (see references for a selection of the relevant literature). Not surprisingly, the evaluation results differ by wide margins and the issue ? economic effects of alternate integration directions ? is excessively politicised. Ukraine has so far rejected a full-fledged BRK-CU membership and has instead acquired ?observer status?. Apart from tricky geo-political aspects, important reasons for Ukraine?s reluctant position are its WTO-related commitments and questions of BRK-CU compatibility with the prospective AA/DCFTA with the EU. While there seems to be little (economic) justification for Russia prompting Ukraine to join the BRK-CU (the economic impacts on Russia are rather small, partly owing to its size), for Ukraine, on the other hand, the economic (and other) linkages to Russia are rather important.

Box 1
\cast{What is the content of the EU-Ukraine DCFTA?}

The EU-UA DCFTA represents part of the Association Agreement and consists of 15 Chapters, 14 Annexes and 3 protocols ? altogether more than 900 pages of text published in November 2012, with few experts ever having probably actually read it. According to Chapter 1 (Market Access for Goods), the vast majority of customs duties (99.1\% by Ukraine and 98.1\% by the EU) will be removed as soon as the Agreement enters into force after the ratification process is completed. A few sectors will obtain transition periods for the removal of customs duties (e.g. the automotive sector in Ukraine for 15 years and some agriculture products in the EU for up to 10 years); WTO rules will be generally applied to non-tariffs barriers. According to EC estimates, Ukrainian exporters will save EUR 487 million annually due to reduced EU import duties, while Ukraine will remove around EUR 390 million in duties on imports from the EU.\pozn{The net effect on Ukraine would thus be a gain of some EUR 100 million. In contrast, at a recent conference in Kharkiv, Sergey Glazyev predicted a deterioration of Ukraine?s trade balance in the event of DCFTA signature by USD 5 billion owing to the abolishment of customs duties on 75\% of imports (see www.gazeta.ru from 1 November, 2013).} Ukraine will progressively adapt its technical regulations and standards to those of the EU.\pozn{Ibid., Chapter 3, Technical barriers to trade. There is no available EC estimate for Ukraine?s acquis takeover costs but, according to Ukrainian sources, these costs are doubtless considerable (see also Dreyer, 2012). Commissioner F�le, in his speech on 11 October, 2013, mentioned the intention to help with an ?indicative amount of EUR 186 million?. For an earlier experience of NMS see Havlik (2003).} Chapter 6 (which deals with services) aims at the expansion of the EU internal market ?once Ukraine effectively implements the EU-acquis?. Similar wording is used in relation to financial services, telecom, postal and maritime services. Chapter 8 (Public procurement) provides exceptions for the defence sectors in both Ukraine and the EU. For the first time, Ukraine?s DCFTA includes specific provisions on trade-related energy issues (Chapter 11; Ukraine is already a member of the Energy Community Treaty, which imposes an obligation to implement the EU energy acquis on electricity and gas). These include rules on pricing, the prohibition of dual pricing and transport interruption to third countries, as well as rules on non-discriminatory access to the exploration and production of hydrocarbons.\pozn{Note the similar above-quoted conditionality required by the IMF.} Importantly, Protocol I of the DCFTA deals with rules of origin and defines the ?economic nationality? of products needed to determine the duties applicable to traded goods.\pozn{The latter is one of Russia?s major complaints regarding the incompatibility of the DCFTA and the BRK-CU (and the existing CIS-wide FTA where Ukraine is a member) and is used as an argument for the erection of trade barriers.} Future EU-Ukraine relations will include EU-Ukraine summits and the Association Council with the power to take binding decisions. Last but not least, Article 39 of the agreement explicitly stipulates that the DCFTA ?shall not preclude the maintenance or establishment of customs unions, free trade areas or arrangements for frontier traffic except insofar as they conflict with trade arrangements provided for in this agreement? and consultations regarding these matters will take place within the Trade Committee.

Source: European Commission, DG Trade and Industry. For the English version of the text, see 
EU Ukraine Association Agreement English - 2012\_11\_19\_EU\_Ukraine\_Association\_Agreement\_English.pdf published on 19 November, 2012 (a concise summary was published on the European Commission DG Trade website on 26 February, 2013).

Notwithstanding the above incompatibilities, which would have to be re-negotiated and would doubtless leave room for compromises, the polarisation of Russian and EU standpoints regarding these issues is not only endangering future Russian-EU relations, but is also counterproductive with respect to Ukraine, which remains sandwiched between the two and would be ultimately adversely affected by EU-Russia frictions the most. As far as foreign trade volumes are concerned, Russia and the EU are of about the same importance for Ukraine: Ukraine?s exports to each of the destinations amounted to some USD 17 billion in 2012. Russia accounted for 26\% of Ukraine?s exports and the BRK-CU (together with Belarus and Kazakhstan) for 33\% of Ukraine?s exports in 2012. The enlarged EU(28) accounted for 25\% of Ukraine?s exports in 2012 (see Annex for additional trade statistics). As regards imports, the situation is similar: 32\% of Ukraine?s imports originated from Russia in 2012 (and more than 40\% from the BRK-CU), whereas imports from the EU(28) accounted for 31\% of the total. However, there are important structural aspects of Ukraine?s trade to either destination: the structure of exports to Russia is more ?advanced?, since Ukraine?s exports of transport equipment and machinery play a much bigger role. Some Ukrainian estimates reckon with an additional export and GDP growth potential from exports to Russia, especially in aircraft, shipbuilding and railway machinery industries.\pozn{Calculations by L. Shinkaruk, Institute for Economics and Forecasting, National Academy of Sciences of Ukraine (mimeo).}

With respect to the EU, Ukraine?s exports are specialised on vegetable products, mineral products (partly refined from Russian oil imports) and base metals. Ukraine?s imports from Russia are traditionally dominated by mineral products, whereas imports from the EU consist mostly of chemicals, machinery and transport equipment (Figure 1). 

Russia and the EU are thus nearly equally important trading partners for Ukraine. From a purely trade importance point of view the either-or decision regarding the direction of Ukraine?s trade integration is rather meaningless: both directions are important. Restricted access to the Russian market ? if trade barriers are introduced by Russia as a punishment in case of Ukraine?s ?European integration? choice ? would hit a more advanced part of Ukraine?s economy (located largely in the eastern part of the country) immediately and disproportionally, irrespective of the fact that a large part of these exports may represent remnants of cooperation links from the Soviet past (and are largely not competitive on EU markets). A BRK-CU-oriented integration of Ukraine would help to maintain and develop existing technological cooperation linkages, though probably without much modernisation and restructuring pressures (unless Russia itself embarks on a more radical reform path). On the other hand, the implementation of the AA/DCFTA with the EU would bring benefits to Ukraine only in the medium and long run ? especially regarding the expected pressure on modernisation and reforms which would eventually lead to a significant restructuring of the Ukrainian economy and higher FDI inflows. There is little doubt that the EU, as a more developed economy, would introduce more competition, modernisation and reform pressures on Ukraine; the EU market is also much bigger than the Russian one. 

\popis{Figure 1 Structure of Ukraine?s foreign trade (in \% of total, 2012)}



Note: 
I  Live animals, animal products; 
II  Vegetable products
III  Animal or vegetable fats, oils, waxes, prepared edible fats
IV  Prepared foodstuffs, beverages, tobacco and substitutes
V  Mineral products
VI  Products of the chemical or allied industries
VII  Plastics and articles thereof, rubber and articles thereof
VIII  Raw hides and skins, leather, furs, etc.
IX  Wood and articles of wood, wood charcoal, cork, etc.
X  Pulp wood, paper or paperboard (incl. recovered) and articles
XI  Textiles and textile articles
XII  Footwear, headgear, umbrellas, walking sticks, etc.
XIII  Articles of stone, plaster, cement, ceramic products, glassware
XIV  Natural or cultured pearls, precious stones and metals, etc.
XV  Base metals and articles of base metal
XVI  Machinery, mech. appliances, electr. equipment
XVII  Vehicles, aircraft, vessels and associated transport equipment
XVIII  Optical, measuring, medical instr., clocks, musical instr., etc.
XX  Miscellaneous manufactured articles
Source: State Statistics Committee of Ukraine; own calculations.

As far as customs tariffs are concerned, Ukraine and Russia have a formal free trade agreement (with some important exceptions for agricultural products such as sugar) while in trade with the EU, 70.6\% of the value of Ukrainian agricultural products and 90.8\% of the value of non-agricultural products were already exported duty-free in 2011. Russia faced similar tariff protection in the EU for agriculture products like Ukraine while nearly all Russian non-agricultural exports to the EU were duty free (in value; in terms of the number of duty-free tariff lines, Ukraine?s agricultural products face greater trade barriers in the EU ? see Table 1 and Annex). Ukraine?s (as well as Russia?s) exports face the highest tariff protection in dairy products, cereals, sugar, beverages and tobacco, whereas industrial products generally enjoy more tariff protection in both Ukraine and Russia. In fact, average final bound duties in both Ukraine and Russia are very similar (except for animal products, beverages and tobacco, and wood and paper where Russian tariffs are higher and the harmonisation of tariff lines should not, given the will to negotiate, pose too big a problem ? with the above-quoted few exceptions, see Table 1).


\popis{Table 1 Tariffs and imports by product groups, Ukraine}

 
Final bound duties
MFN applied duties 
Imports 
Differences in final bound duties AVG
Product groups
AVG
Duty-free
Max
Binding
AVG
Duty-free
Max
Share
Duty-free
EU-RU
EU-UA
RU-UA
 
 
in \%
 
in \%
 
in \%
 
in \%
 in \%



Animal products
13.0 
        0
       20
  100
11.0  
    9.0
       20
    0.5
     15.0
0.3
10.4
10.1
Dairy products
10.0 
        0
       10
  100
10.0  
      0
       10
    0.2
        0
39.8
44.7
4.9
Fruit, vegetables, plants
13.1 
     10.2
       20
  100
9.9  
   18.9
       20
    1.4
     54.6
1.5
-2.9
-4.4
Coffee, tea
5.8 
     35.4
       20
  100
5.8  
   35.4
       20
    1.3
     42.0
-0.2
0.4
0.6
Cereals \& preparations
12.7 
      3.3
       20
  100
12.6  
    3.8
       20
    0.9
     27.1
12.1
9.5
-2.6
Oilseeds, fats \& oils
10.7 
     11.0
       30
  100
8.3  
   20.1
       30
    0.9
     89.9
-1.5
-5.1
-3.6
Sugars and confectionery
17.5 
      0.6
       50
  100
17.5  
      0
       50
    0.3
        0
18.3
13.5
-4.8
Beverages \& tobacco
7.9 
     25.7
       64
  100
12.2  
   26.2
      424
    1.2
     23.9
-2.3
13.4
15.7
Cotton
1.4 
     40.0
        5
  100
1.4  
   40.0
        5
    0.0
     61.3
0
-1.4
-1.4
Other agricultural products
7.6 
     23.9
       20
  100
5.5  
   45.2
       20
    0.5
     19.3
-1.2
-3.5
-2.3
Fish \& fish products
3.7 
     61.7
       20
  100
2.6  
   68.2
       20
    0.7
     68.0
3.4
7.2
3.8
Minerals \& metals
4.5 
     42.4
       20
  100
3.0  
   47.6
       20
   32.8
     79.0
-6
-2.5
3.5
Petroleum
1.5 
     72.0
       10
  100
0.9  
   84.3
       10
   13.7
     97.3
-3
0.5
3.5
Chemicals
5.1 
     16.1
       10
  100
3.2  
   39.3
       65
   12.7
     55.4
-0.6
-0.5
0.1
Wood, paper, etc.
0.4 
     95.8
       10
  100
0.3  
   95.8
       10
    3.1
     99.1
-7
0.5
7.5
Textiles
4.1 
     33.7
       13
  100
3.8  
   35.6
       13
    2.1
     25.6
-1.3
2.4
3.7
Clothing
11.4 
      1.0
       12
  100
11.3  
    1.1
       12
    0.6
      0.1
-0.3
0.1
0.4
Leather, footwear, etc.
7.2 
     14.9
       25
  100
5.4  
   27.0
       25
    1.9
     20.3
-2.2
-3
-0.8
Non-electrical machinery
4.2 
     38.7
       12
  100
2.1  
   51.3
       10
    8.8
     62.4
-4.1
-2.5
1.6
Electrical machinery
5.3 
     33.0
       25
  100
3.8  
   39.1
       25
    6.8
     34.2
-3.8
-2.9
0.9
Transport equipment
7.5 
     15.8
       20
  100
5.1  
   39.6
       20
    7.5
     21.4
-4.8
-3.4
1.4
Manufactures, n.e.s.
6.4 
     31.9
       25
  100
5.5  
   32.0
       25
    2.1
     68.6
-5.9
-3.9
2
\zdroj{Source: WTO; own calculations.}


\podnadpis{Conclusion}
Cooperation and integration, not confrontation
The earlier (both positive and negative) integration experiences of the new EU Member States (NMS) may provide a useful reference point for Ukraine. NMS trade integration with the EU advanced rapidly after they had signed association agreements and inflows of FDI to the region had already accelerated before EU accession. FDI inflows have brought new technologies, higher quality standards, and better know-how in management and marketing (Hunya, 2008). Last but not least, FDI inflows have facilitated access to EU markets and fostered modernisation; they even contributed to a revival of intra-NMS trade (Richter, 2011). FDI-induced modernisation was also crucial in raising the energy efficiency of the recipient countries? economies (which remains an important challenge for Ukraine ? see Astrov et al., 2012). In this way, the former COMECON countries have successfully restructured their industrial sector, which in many cases became competitive on the European scale and has been gaining global market shares (Havlik, 2008). But the experience of the NMS in the recent crisis has also taught important lessons regarding the negative effects of capital flows and integration ? neither being a panacea with respect to growth and convergence (see, for example, Gligorov et al., 2012).

In the case of Ukraine ? unlike in the above-mentioned NMS countries ? one important factor behind the success restructuring story, namely the ?carrot? of prospective EU membership, is missing and is unlikely to be in place any time soon. Theoretically, Ukraine (just as Russia) could still try to emulate these developments via closer EU integration ? even without a formal accession anchor, as the Baltic States did in the early 1990s.\pozn{It is questionable as to whether this incentive is sufficient for truly sustained reform efforts. WTO membership is definitely not a sufficient ?reform anchor? ? see O. Havrylyshyn in Grinberg et al. (2008).
} The latter does not rule out that Ukraine maintains close economic links with Russia, e.g. via a preservation of the current free trade regime (albeit with exemptions and limitations). The BRK-CU members ? and first of all Russia ? should also advance their integration with the enlarged EU, at least to the stage of a free trade area. Closer EU-BRK-CU integration which would include Ukraine is a potentially preferred option in future, and would, if accompanied by a parallel integration of other EaP countries, lay the foundation for a broader Pan-European Economic Space and wider Eurasian integration ?from Lisbon to Vladivostok?. This could be part of the new inclusive strategy for the EU Eastern Partnership which would refrain from strategic rivalry with Russia and revitalise the Partnership for Modernisation, especially in order to avoid trade wars and the raising of new walls in Europe (Samson, 2002; Havlik, 2010; Emerson, 2011a; Havlik, 2013; Wisniewski, 2013; etc.).

In summary, both Russia and the EU should abstain from counterproductive geopolitical games over influence in the EaP region which would have adverse consequences, especially for the EaP countries concerned. EU-Russia negotiations should not be about Ukraine or other EaP countries but should involve the latter in the process. All parties should also continue/resume FTA negotiations ? perhaps with a lesser and selective focus on costly harmonisations of norms and regulations. Last but not least, progress on visa liberalisation procedures and other confidence-building measures should be decisively speeded up and here it is the EU which should deliver. Apart from confidence building measures, closer integration of the enlarged EU, Russia and the Eastern Partnership countries ? from ?Lisbon to Vladivostok? ? would boost trade and investment, thus fostering badly needed economic growth and stability in Europe.

\cast{References}
Alili, Z., T. Abbasov, D.N. Chang and M. Hoyt (2013): Accession to the Customs Union: Shaping the Strategy for Azerbaijan. CESD, Baku.
Astrov, V. (2011): The EU and Russia: both important for Ukraine, Eastern Partnership Community, 23 May. http://www.easternpartnership.org/community/debate/eu-and-russia-both-important-ukraine
Astrov, V., P. Havlik and O. Pindyuk (2012): Trade Integration in the CIS: Alternate Options, Economic Effects and Policy Implications for Belarus, Kazakhstan, Russia and Ukraine, wiiw Research Report 381, Vienna: Vienna Institute for International Economic Studies, wiiw.
ATF Bank (2010): Customs Union: no big inflation shock, but efforts needed to offset impact on non-resources sectors, 4, April.
Baldwin, R. (1994): Towards an Integrated Europe, London: CEPR.
Boss, H. and P. Havlik (1994): Slavic (dis)union: consequences for Russia, Belarus and Ukraine, Economics of Transition, Oxford University Press, 2(2), 233--254.
Cameiro, F.G. (2013): What Promises Does the Eurasian Customs Union Hold for the Future? Economic Premise, The World Bank, No. 108, February.
Dabrowski, M. and M. Maliszewska (eds.) (2011): EU Eastern Neighborhood, Berlin: Springer.
Dabrowski, M. and S.Taran (2012): The Free Trade Agreement between the EU and Ukraine: Conceptual Background, Economic Context and Potential Impact, CASE Network Studies and Analyses, No. 437. 
De Gucht, K. (2011): EU-Ukraine trade negotiations: a pathway to prosperity, INTA Committee Workshop, Brussels, 20 October.
Dreyer, I. (2012): What economic benefit to expect from DCFTAs? Visegrad Group and Germany Policy Makers Seminar, Ministry of Foreign Affairs, Prague, November.
EBRD (2012): Transition Report 2012. Integration Across Borders. Chapter 4. London.
ECORYS/CASE (2013): Trade Sustainability Impact Assessment in support of negotiations of a DCFTA between the EU and the Republic of Armenia. European Commission, Rotterdam, DG Trade, September.
Efremenko, D.V. (2012): Waiting for a Storm. Russian Foreign Policy in the Era of Change, Russia in Global Affairs, 2, April-June.
Efremenko, D.V. (2013): Life after Vilnius. A New Geopolitical Configuration for Ukraine, Russia in Global Affairs, No. 3, pp. 133-146. 
EDB Centre for Integration Studies (2012): Ukraine and the Customs Union. Comprehensive Assessment of the Macroeconomic Effects of Various Forms of Deep Economic Integration of Ukraine and the Member States of the Customs Union and the Common Economic Space, St. Petersburg. 
EDB Centre for Integration Studies (2013), Monitoring of Mutual Investments. EDB Centre for Integration Studies? Report no. 15. September. EDB: St. Petersburg.
EDB Centre for Integration Studies (2013): The Customs Union and Neighbouring Countries: Mechanisms and Instruments of Mutually Beneficial Partnership. EDB Centre for Integration Studies? Report no. 11. ??rch. EDB: St. Petersburg.
EDB Centre for Integration Studies (2013: Economic and Technological Cooperation by Sectors SES and Ukraine. EDB Centre for Integration Studies? Report no. 18. November. EDB: St. Petersburg. Emerson, M. (2005): EU?Russia ? the Four Common Spaces and the Proliferation of the Fuzzy, CEPS Policy Brief, Brussels, May.
Emerson, M. (2011a): Review of the review ? of the European Neighbourhood Policy, CEPS European Neighbourhood Watch, 71, May.
Emerson, M. (2011b): The Timoshenko case and the rule of law in Ukraine, CEPS European Neighbourhood Watch, 73, July.
Emerson, M. and E. Vinokurov (2009): Optimisation of Central Asian and Eurasian Trans-Continental Land Transport Corridors. EUCAM, Working paper 07, December.
European Commission (2010): Taking stock of the European Neighbourhood Policy, Communication from the Commission to the European Parliament and the Council. Brussels: European Commission.
European Economy (2011): The EU?s Neighbouring Economies: Coping with new challenges, Occasional papers 86, DG ECFIN, November.
Falvey, R. and N. Foster-McGregor (2013): On the Trade and Price Effects of Preferential Trade Agreements. wiiw Working Papers, No. 102, May
Francois, J. and M. Manchin (2009): Economic Impact of a Potential Free Trade Agreement (FTA) between the European Union and the Commonwealth of Independent States, Institute for International and Development Economics Discussion Paper 200908-05.
Francois, J., M. Manchin, H. Norberg, O. Pindyuk and P. Tomberger (2013): Reducing Trans-Atlantic Barriers to Trade and Investment, European Commission and CEPR.
F�le, �. (2011): Future prospects for EU enlargement and Neighbourhood policy, Chatham House, 13 January.
F�le, �. (2012): Speech at the conference EU-Nachbarschaft-Der Arabische Fr�hling ein Jahr danach?, Munich, 3 February.
F�le, �. (2013a): EU-Ukraine: In Yalta about progress towards signing the Association Agreement. 10th Yalta Annual Meeting, in Yalta, Ukraine, 20 September, Speech/13/727.
F�le, �. (2013b): EU-Ukraine: Dispelling the Myths About the Association Agreement. International Conference ?The Way Ahead for the Eastern Partnership?, Kyiv, Ukraine, 11 October, Speech/13/808.
F�le, �. (2013c): Speech at the National Round Table on European integration Kyiv, Ukraine, 11 October, Speech/13/810.
Gligorov, V., M. Holzner, M. Landesmann, S. Leitner, O. Pindyuk and H. Vidovic (2012): New Divide(s) in Europe? Current Analyses and Forecasts, wiiw Research Report 9, Vienna: Vienna Institute for International Economic Studies, wiiw.
Grant, Ch. (2011): A New Neighbourhood Policy for the EU, Centre for European Reform Policy Brief, London: CERP.
Grinberg, R., P. Havlik and O. Havrylyshyn (eds) (2008): Economic Restructuring and Integration in Eastern Europe. Experiences and Policy Implications, Baden Baden: Nomos.
Hamilton, C.B. (2005): Russia?s European economic integration. Escapism and realities, EconomicSystems, 29, 294?306.
Havlik, P. (2003): EU Enlargement: Implications for Growth and Competitiveness?, A study commissioned by the Austrian Ministry for Economic Affairs and Labour. wiiw, Vienna, August.
Havlik, P. (2004): Russia, European Union and EU Eastward Enlargement, In: G.Hinteregger and H.G.Heinrich (eds.), Russia ? Continuity and Change, Springer Vienna New York, 363-378.
Havlik, P. (2008): Structural change and trade integration on EU-NIS borders?, In: R.Grinberg et al. (eds.), Economic Restructuring and Integration in Eastern Europe. Experiences and Policy Implications, Baden Baden: Nomos, 119?148.
Havlik, P. (2010): European Energy Security in View of Russian Economic and Integration Prospects, wiiw Research Report 362, Vienna: Vienna Institute for International Economic Studies, wiiw.
Havlik, P. (2013): The European Union and Eurasia: Challenges of Economic Integration. Presentation at the VIII. Eurasian Development Bank Conference on Deepening and Widening of Eurasian Integration. Moscow, November. http://www.eabr.org/general//upload/8\_Conference\_DOC/presentations/Havlik-1.pdf. 
Havlik, P., R. St�llinger, O. Pindyuk, G. Hunya, B. Dachs, C. Lennon, M.P. Ribeiro, J. Ghosh, W. Urban, V. Astrov and E. Christie (2009): EU and BRICs: Challenges and opportunities for European competitiveness and cooperation, Industrial Policy and Economic Reform Papers, 13: http://ec.europa.eu/enterprise/newsroom/cf/\_getdocument.cfm?doc\_id=5586
Havlik, P. et al. (2012): European Neighbourhood ? Challenges and Opportunities for EU Competitiveness, wiiw Research Report 382, Vienna: Vienna Institute for International Economic Studies, wiiw.
Havrylyshyn, O. (2008): Structural change in transition 1990-2005: A comparison of New Member States and selected NIS countries, In: R.Grinberg et al. (eds.), Economic Restructuring and Integration in Eastern Europe. Experiences and Policy Implications, Baden Baden: Nomos, 17?48.
Hunya, G. (2008): FDI in the new EU borderland, in: R. Grinberg et al. (2008), Economic Restructuring and Integration in Eastern Europe. Experiences and Policy Implications, Baden Baden: Nomos, 73?94.
Institute for Economic Research and Policy Consulting (2011): Ukraine?s Trade Policy Choice: Pros and cons of different regional integration options, Kiev: IERPC.
Institute of Economics and Forecasting of the National Academy of Sciences of Ukraine (2011): ??????????? ?????????? ?????? ??????????? ????????? ????????? ????? ??? ???? ??????? ???????? ? ?? ??? ????????? ?? ??????? ????? ?????, ???????? ?? ??????????? (Approximate analytical estimate of economic consequences of FTA with EU or joining the Customs Union of Russia, Kazakhstan, and Belarus), Kyiv, mimeo.
Isakova, A. and A. Plekhanov (2012): Customs Union and Kazakhstan?s Imports. CASE Network Studies and Analyses, No. 442.
Kolesnikova, I. (2013): WTO Accession and Economic Development: Experience of Newly Acceded Countries and Implications for Belarus. Polish Development Cooperation Program of the Polish Ministry of Foreign Affairs, Warszaw. 
Leonard, M. (2011): Europe?s multipolar neighborhood, Carnegie Europe, 30 September.
Libman, A. and E. Vinokurov (2012): Regional Integration and Economic Convergence in the Post-Soviet Space: Experience of a Decade of Growth. Journal of Common Market Studies. Vol. 50. Number 1. 
pp. 112?128.
Linn, J.F. (2004): Economic (Dis)integration Matters: The Soviet Collapse Revised. Paper prepared for the conference on ?Transition in the CIS: Achievements and Challenges? at the Academy for National Economy, Moscow, 13-14 September.
Lissovolik, B. and Y. Lissovolik (2006): Russia and the WTO: The ?Gravity? of Outsider Status, IMF Staff Papers, 53(1), 1-27.
Maliszewska, M., I. Orlova and S. Taran (2009): Deep Integration with the EU and its Likely Impact on Selected ENP Countries and Russia, CASE Network report 88, Warsaw: CASE.
Malynovska, O. (2006): Caught between East and West, Ukraine struggles with its migration policy:http://www.migrationinformation.org/Profiles/display.cfm?ID=365
Miszei, K. (2013): Why European and not Eurasian Integration? IPN CAMPAIGN Moldova. http://ipn.md/en/special 57775\#.
Movchan, V. and R. Giucci (2011): Quantitative Assessment of Ukraine?s Regional Integration Options: DCFTA with European Union vs. Customs Union with Russia, Belarus and Kazakhstan, IRPC Policy Paper Series (PP/05/2011), Berlin/Kiev: IRPC.
Movchan, V. and V. Shportyuk (2012): EU-Ukraine DCFTA: the Model for Eastern Partnership Regional Trade Cooperation. CASE Network Studies and Analyses, No. 445.
Richter, S. (2011): Revival of the Visegrad Countries? Mutual Trade after their EU Accession: a Search for Explanation' (with Neil Foster-McGregor, Gabor Hunya and Olga Pindyuk), wiiw Research Report, No. 372, Vienna, July.
Samson, I. (2002): The Common European Economic Space Between Russia and the EU: An Institutional Anchor for Accelerating Russian Reform, Russian Economic Trends, 11(3).
Tochitskaya, I. (2010): The Customs Union between Belarus, Kazakhstan and Russia: an overview of economic implications for Belarus, CASE Network Studies \& Analyses 405, Warsaw: CASE.
Vinhas de Souza, L. (2011): An initial estimation of the economic effects of the creation of the EurAsEC Customs Union on its members, PREM Network Economic Premise, 47, January.
Vinokurov, E. and A. Libman (2012): Eurasia and Eurasian Integration: Beyond the Post-Soviet Borders. Eurasian Integration Yearbook 2012. EDB: St. Petersburg, pp.80-96.
Vinokurov, E. and A. Libman (2012): Eurasian integration: Challenges of transcontinental regionalism, Basingstoke: Palgrave MacMillan. 
Vinokurov, E. (2013): Pragmatic Eurasianism, Russia in Global Affairs, 2.
Wallace, H. (2009): The European Union and its neighbourhood: time for a rethink, ELIAMEP Thesis, 4, May.
Wisniewski, P.D. (2013): It Is High Time to Start a ?Real Partnership?. Carnegie Moscow Center, November. (http://carnegie.ru/2013/11/20).
World Bank (2013): Doing Business 2014: Doing Business in a More Transparent World. Washington, D.C.: World Bank. (http://www.doingbusiness.org).
wiiw (2013): Handbook of Statistics. Countries in Transition. The Vienna Institute for International Economic Studies, Vienna, November.
Youngs, R. and K. Pishchikova (2013): Smart Geostrategy for the Eastern Partnership. Carnegie Europe, November.



Statistical Annex

Tables from the wiiw Handbook of Statistics: Countries in Transition 2013.

\popis{Table A1 Kazakhstan: Foreign trade by country groupings}


2000
2005
2009
2010
2011
2012*
EUR mn 1)2)






Exports, fob






Total
9319
22371
30977
45387
62929
67249
EU-28
2400
9034
15164
23203
30738
35364
   EU-15
2181
7752
12705
20391
27253
30665
Other countries 3)
6919
13337
15813
22185
32191
31884
Imports, cif






Total
5330
13939
20373
23440
26619
36021
EU-28
1253
3453
5588
5482
5271
7270
   EU-15
1074
2995
4805
4567
4355
6069
Other countries 3)
4077
10486
14785
17958
21348
28752
Trade balance






Total
3989
8432
10604
21947
36310
31227
EU-28
1147
5581
9576
17721
25467
28095
   EU-15
1108
4757
7900
15824
22898
24597
Other countries 3)
2842
2851
1028
4226
10843
3133







Annual growth in \%






Exports, fob






Total
72.6
38.3
-36.0
46.5
38.6
6.9
EU-28
62.1
59.5
-27.0
53.0
32.5
15.1
   EU-15
87.0
52.7
-32.1
60.5
33.7
12.5
Other countries 3)
76.6
26.9
-42.7
40.3
45.1
-1.0
Imports, cif






Total
58.6
35.5
-20.9
15.1
13.6
35.3
EU-28
26.9
22.6
-4.3
-1.9
-3.8
37.9
   EU-15
29.1
25.9
-3.2
-5.0
-4.6
39.3
Other countries 3)
71.8
40.4
-25.7
21.5
18.9
34.7







Shares in \%






Exports, fob






Total
100.0
100.0
100.0
100.0
100.0
100.0
EU-28
25.8
40.4
49.0
51.1
48.8
52.6
   EU-15
23.4
34.7
41.0
44.9
43.3
45.6
Other countries 3)
74.2
59.6
51.0
48.9
51.2
47.4
Imports, cif






Total
100.0
100.0
100.0
100.0
100.0
100.0
EU-28
23.5
24.8
27.4
23.4
19.8
20.2
   EU-15
20.1
21.5
23.6
19.5
16.4
16.8
Other countries 3)
76.5
75.2
72.6
76.6
80.2
79.8
1) Officially registered trade.
2) Values in EUR converted from USD to NCU to EUR at the average official exchange rate.
3) Refers to total minus EU-28 from 2000.

\popis{Table A2 Russia: Foreign trade by country groupings}


2000
2005
2009
2010
2011
2012*
EUR mn 1)






Exports, fob






Total
111449
193709
216560
299354
371071
408182
EU-28
60780
111619
116080
160210
192189
216319
   EU-15
39870
80255
88564
121657
142915
164148
Other countries 2)
50668
82090
100480
139143
178882
191863
Imports, cif






Total
36613
79190
120136
172579
219576
246447
EU-28
14617
35375
53962
71947
91606
96044
   EU-15
12044
29283
43287
56998
74154
79421
Other countries 2)
21996
43815
66174
100632
127970
150403
Trade balance






Total
74836
114519
96424
126775
151495
161735
EU-28
46164
76245
62119
88263
100583
120275
   EU-15
27827
50972
45278
64659
68761
84727
Other countries 2)
28672
38275
34306
38511
50912
41460







Annual growth in \%






Exports, fob






Total
63.0
32.6
-32.0
38.2
24.0
10.0
EU-28
80.1
46.2
-36.4
38.0
20.0
12.6
   EU-15
71.0
51.2
-34.2
37.4
17.5
14.9
Other countries 2)
46.3
17.7
-26.2
38.5
28.6
7.3
Imports, cif






Total
28.9
30.3
-34.0
43.7
27.2
12.2
EU-28
17.9
27.8
-32.0
33.3
27.3
4.8
   EU-15
14.9
28.0
-32.3
31.7
30.1
7.1
Other countries 2)
37.4
32.3
-35.5
52.1
27.2
17.5







Shares in \%






Exports, fob






Total
100.0
100.0
100.0
100.0
100.0
100.0
EU-28
54.5
57.6
53.6
53.5
51.8
53.0
   EU-15
35.8
41.4
40.9
40.6
38.5
40.2
Other countries 2)
45.5
42.4
46.4
46.5
48.2
47.0
Imports, cif






Total
100.0
100.0
100.0
100.0
100.0
100.0
EU-28
39.9
44.7
44.9
41.7
41.7
39.0
   EU-15
32.9
37.0
36.0
33.0
33.8
32.2
Other countries 2)
60.1
55.3
55.1
58.3
58.3
61.0
1) Values in EUR converted from USD to NCU to EUR at the average official exchange rate.
2) Refers to total minus EU-28 from 2000.




\popis{Table A3 Ukraine: Foreign trade by country groupings}


2000
2005
2009
2010
2011
2012*
EUR mn 1)






Exports, fob






Total
15764.6
27455.0
28457.9
38729.2
49129.8
53536.7
EU-28
5215.2
8256.5
6820.9
9858.6
12945.4
13321.2
   EU-15
2811.6
4578.1
3906.7
5474.5
6787.7
7371.1
Other countries 2)
10549.4
19198.5
21637.0
28870.6
36184.3
40215.5
Imports, cif






Total
15097.7
28985.3
32571.0
45763.8
59340.2
65867.2
EU-28
4378.8
9794.8
11067.9
14428.9
18536.3
20404.6
   EU-15
3116.9
6755.8
7225.0
8921.6
11938.4
13168.3
Other countries 2)
10718.9
19190.5
21503.1
31334.9
40803.9
45462.6
Trade balance






Total
667.0
-1530.3
-4113.1
-7034.6
-10210.4
-12330.5
EU-28
836.4
-1538.3
-4247.1
-4570.3
-5590.9
-7083.5
   EU-15
-305.3
-2177.7
-3318.3
-3447.1
-5150.6
-5797.2
Other countries 2)
-169.5
8.0
133.9
-2464.3
-4619.5
-5247.1







Annual growth in \%






Exports, fob






Total
44.8
4.4
-37.8
36.1
26.9
9.0
EU-28
51.4
-7.3
-45.4
44.5
31.3
2.9
   EU-15
41.2
-4.7
-40.7
40.1
24.0
8.6
Other countries 2)
41.7
10.4
-35.0
33.4
25.3
11.1
Imports, cif






Total
35.6
24.2
-44.3
40.5
29.7
11.0
EU-28
34.6
27.3
-44.0
30.4
28.5
10.1
   EU-15
38.2
23.7
-42.7
23.5
33.8
10.3
Other countries 2)
36.0
22.7
-44.4
45.7
30.2
11.4







Shares in \%






Exports, fob






Total
100.0
100.0
100.0
100.0
100.0
100.0
EU-28
33.1
30.1
24.0
25.5
26.3
24.9
   EU-15
17.8
16.7
13.7
14.1
13.8
13.8
Other countries 2)
66.9
69.9
76.0
74.5
73.7
75.1
Imports, cif






Total
100.0
100.0
100.0
100.0
100.0
100.0
EU-28
29.0
33.8
34.0
31.5
31.2
31.0
   EU-15
20.6
23.3
22.2
19.5
20.1
20.0
Other countries 2)
71.0
66.2
66.0
68.5
68.8
69.0
1) Values in EUR converted from USD to NCU to EUR at the average official exchange rate.
2) Refers to total minus EU-28 from 2000.




Table A4 Kazakhstan: Exports to top thirty partners



2000
2005
2009
2010
2011
2012*








Total exports, fob, EUR mn 1)
9319.0
22370.9
30977.2
45387.1
62928.6
67248.6
Shares in \% (ranking in 2012)






Italy
1
10.41
15.05
15.48
15.89
17.17
17.77
China
2
7.65
8.70
13.63
16.79
18.60
16.46
Netherlands
3
2.57
3.15
5.15
6.90
7.58
8.43
Russia
4
19.87
10.51
8.21
9.48
7.99
7.09
France
5
0.18
9.57
7.83
7.36
6.18
6.52
Austria
6
0.01
0.00
2.77
4.20
4.43
5.73
Switzerland
7
5.15
19.78
6.18
2.05
5.66
5.69
Canada
8
0.08
1.90
3.21
4.06
3.00
3.56
Romania
9
0.01
1.65
1.95
2.13
2.59
3.51
Turkey
10
0.71
0.56
1.83
2.05
2.94
3.13
Ukraine
11
2.88
0.72
2.98
1.11
3.05
2.76
United Kingdom
12
2.58
1.15
2.86
2.30
1.85
1.94
Poland
13
0.64
1.32
1.93
2.02
1.49
1.87
Israel
14
.
.
2.60
2.12
1.62
1.78
Germany
15
6.25
1.47
2.08
2.90
1.84
1.61
Uzbekistan
16
1.51
0.87
2.06
1.82
1.35
1.36
Portugal
17
.
1.14
0.64
1.22
1.30
1.18
Spain
18
0.07
1.67
1.34
1.53
1.30
0.77
Greece
19
0.01
0.50
1.26
1.65
0.66
0.76
Kyrgyzstan
20
0.66
0.81
0.90
0.70
0.58
0.74
Iran
21
2.31
3.18
2.96
1.81
1.23
0.70
Japan
22
0.11
0.49
0.57
0.89
1.19
0.64
Finland
23
0.79
0.64
1.04
0.45
0.67
0.60
Tajikistan
24
0.60
0.54
0.56
0.43
0.41
0.54
United States
25
2.38
2.39
1.42
1.46
1.17
0.46
Bulgaria
26
0.02
0.00
0.42
0.28
0.55
0.41
Azerbaijan
27
0.53
0.46
0.21
0.57
0.27
0.40
Afghanistan
28
0.66
0.59
0.95
0.60
0.38
0.34
Cyprus
29
0.02
1.03
.
0.01
0.10
0.26
Korea Republic
30
0.41
0.67
0.30
0.39
0.32
0.25
1) Officially registered trade.




\popis{Table A5 Russia: Exports to top thirty partners}



2000
2005
2009
2010
2011
2012*








Total exports, fob, EUR mn 
111449
193709
216560
299354
371071
408182
Shares in \% (ranking in 2012)






Netherlands
1
4.22
10.19
12.07
13.59
12.13
14.64
China
2
5.09
5.40
5.53
5.12
6.78
6.81
Germany
3
8.95
8.17
6.20
6.46
6.61
6.78
Italy
4
7.03
7.89
8.32
6.92
6.32
6.18
Turkey
5
3.00
4.49
5.43
5.12
4.91
5.23
Ukraine
6
4.87
5.14
4.59
5.83
5.90
5.18
Belarus
7
5.40
4.19
5.54
4.55
4.82
4.68
Poland
8
4.32
3.57
4.14
3.76
4.14
3.79
Japan
9
2.68
1.55
2.40
3.23
2.83
2.97
Kazakhstan
10
2.18
2.71
3.03
2.69
2.73
2.87
United Kingdom
11
4.53
3.43
3.01
2.85
2.71
2.86
Korea Republic
12
0.94
0.98
1.88
2.63
2.59
2.63
United States
13
4.50
2.62
3.03
3.10
3.18
2.47
Finland
14
3.01
3.17
3.04
3.06
2.55
2.29
Switzerland
15
3.74
4.46
2.06
2.20
2.22
2.05
France
16
1.85
2.53
2.89
3.13
2.88
2.01
Latvia
17
1.58
0.49
1.37
1.48
1.43
1.70
India
18
1.05
0.96
1.97
1.61
1.18
1.51
Belgium
19
0.73
1.02
1.34
1.24
1.45
1.30
Hungary
20
2.33
2.07
1.29
1.35
1.50
1.27
Sweden
21
1.68
0.96
1.06
0.90
0.99
1.18
Slovakia
22
2.06
1.32
0.98
1.15
1.37
1.17
Greece
23
1.23
0.80
0.77
0.72
0.91
1.13
Spain
24
1.04
1.17
0.96
1.02
1.19
1.09
Lithuania
25
2.01
1.66
1.13
0.89
1.40
1.03
Czech Republic
26
1.69
1.58
1.47
1.39
1.05
1.00
Bulgaria
27
0.57
0.79
0.73
0.86
0.68
0.83
Estonia
28
1.20
0.88
0.38
0.43
0.55
0.70
Taiwan
29
0.39
0.60
0.26
0.45
0.41
0.63
Egypt
30
0.44
0.43
0.60
0.48
0.45
0.61




\popis{Table A6 Ukraine: Exports to top thirty partners}



2000
2005
2009
2010
2011
2012*








Total exports, fob, EUR mn
15764.6
27455.0
28457.9
38729.2
49129.8
53536.7
Shares in \% (ranking in 2012)






Russia
1
24.12
21.88
21.40
26.12
28.98
25.62
Turkey
2
5.96
5.92
5.36
5.89
5.48
5.36
Egypt
3
1.52
2.33
2.55
0.43
1.95
4.21
Poland
4
2.87
2.95
3.04
3.48
4.09
3.74
Italy
5
4.38
5.53
3.09
4.69
4.44
3.60
Kazakhstan
6
0.53
1.95
3.57
2.53
2.72
3.57
India
7
1.15
2.15
2.90
0.97
3.31
3.33
Belarus
8
1.87
2.60
3.17
3.69
2.81
3.27
China
9
4.32
2.08
3.61
0.91
3.19
2.58
Germany
10
5.09
3.75
3.14
2.92
2.58
2.39
Spain
11
1.12
1.68
1.44
0.80
1.42
2.24
Hungary
12
2.25
2.01
1.84
1.67
1.96
2.19
Lebanon
13
0.42
0.30
1.75
0.58
1.99
2.07
Iran
14
0.62
1.69
1.90
0.55
1.65
1.69
United States
15
4.98
2.79
0.63
1.58
1.63
1.47
Saudi Arabia
16
0.25
1.13
1.26
0.16
1.19
1.35
Netherlands
17
0.95
1.51
1.50
1.10
1.22
1.21
Moldova
18
1.21
1.98
1.75
1.39
1.28
1.20
Israel
19
0.73
0.85
0.99
0.31
0.75
1.16
Azerbaijan
20
0.28
0.85
1.38
1.19
1.04
1.11
Czech Republic
21
1.30
1.10
0.86
1.22
1.23
1.03
Slovakia
22
1.58
1.48
1.09
1.11
1.23
0.98
Syria
23
1.10
1.96
1.90
0.36
1.35
0.84
Bulgaria
24
2.62
1.59
1.00
0.88
1.10
0.83
Romania
25
1.13
1.43
0.80
1.37
1.39
0.80
United Kingdom
26
0.94
1.05
0.87
0.99
0.71
0.80
France
27
0.77
0.58
1.11
0.93
0.83
0.80
Georgia
28
0.26
0.58
1.00
1.03
0.96
0.79
Jordan
29
0.31
0.53
1.20
0.20
0.66
0.78
Turkmenistan
30
1.02
0.55
0.82
0.41
0.35
0.77





\popis{Table A7 Kazakhstan: Imports from top thirty partners}



2000
2005
2009
2010
2011
2012*








Total imports, cif, EUR mn 1)
5329.9
13939.0
20372.8
23440.1
26618.5
36021.2
Shares in \% (ranking in 2012)






Russia
1
48.40
37.98
31.32
39.38
41.38
36.59
China
2
3.00
7.21
12.56
12.73
13.55
16.08
Germany
3
6.66
7.50
7.19
5.93
5.62
8.26
Ukraine
4
1.61
4.87
7.50
4.37
4.68
6.33
United States
5
5.50
6.94
4.90
4.24
4.63
4.60
Italy
6
3.09
3.91
6.74
5.10
3.09
2.11
Korea Republic
7
1.66
1.48
1.32
1.69
1.68
2.09
Japan
8
2.09
3.45
2.24
1.80
1.74
1.97
Turkey
9
2.86
2.30
2.01
1.99
1.97
1.74
Uzbekistan
10
1.40
1.47
1.07
1.52
2.08
1.74
Belarus
11
0.78
1.20
1.29
1.70
1.60
1.43
France
12
1.50
1.68
1.62
1.60
1.86
1.41
United Kingdom
13
4.43
2.44
2.47
2.34
1.42
1.30
Poland
14
1.16
1.14
1.48
1.22
1.06
1.04
Kyrgyzstan
15
0.60
0.68
0.41
0.53
0.65
0.79
India
16
0.91
0.58
0.55
0.64
0.66
0.72
Czech Republic
17
0.67
0.55
0.63
0.54
0.44
0.70
Brazil
18
0.55
0.96
0.71
0.75
0.92
0.65
Netherlands
19
1.30
0.81
1.12
0.97
0.79
0.62
Austria
20
0.36
0.90
0.89
0.71
0.60
0.58
Sweden
21
0.51
1.51
0.92
0.67
0.84
0.54
Finland
22
1.14
1.14
1.09
0.67
0.67
0.54
Spain
23
0.18
0.44
0.42
0.32
0.40
0.50
Switzerland
24
1.08
1.16
0.55
0.58
0.42
0.48
Canada
25
0.46
0.73
0.87
0.70
0.47
0.45
Belgium
26
0.66
0.83
0.55
0.57
0.48
0.44
Lithuania
27
0.19
0.16
0.38
0.35
0.27
0.41
Turkmenistan
28
0.86
0.29
0.22
0.03
0.18
0.39
Hungary
29
0.51
0.40
0.35
0.41
0.44
0.31
Ireland
30
.
.
0.23
0.27
0.28
0.27
1) Officially registered trade.





\popis{Table A8 Russia: Imports from top thirty partners}



2000
2005
2009
2010
2011
2012*








Total imports, cif, EUR mn
36613
79190
120136
172579
219576
246447
Shares in \% (ranking in 2012)






China
1
2.80
7.36
13.62
17.02
15.78
15.40
Germany
2
11.51
13.45
12.69
11.66
12.32
12.09
Ukraine
3
10.78
7.92
5.46
6.14
6.58
5.68
Japan
4
1.69
5.91
4.33
4.48
4.91
4.95
United States
5
7.95
4.62
5.48
4.85
4.77
4.83
France
6
3.50
3.72
5.04
4.39
4.34
4.35
Italy
7
3.58
4.47
4.72
4.39
4.38
4.24
Belarus
8
10.95
5.79
4.01
4.35
4.48
3.56
Kazakhstan
9
6.49
3.27
2.21
1.94
2.34
2.72
United Kingdom
10
2.54
2.81
2.12
2.00
2.35
2.59
Korea Republic
11
1.06
4.06
2.91
3.18
3.79
2.17
Turkey
12
1.03
1.75
1.92
2.13
2.08
2.16
Poland
13
2.11
2.78
2.52
2.55
2.18
2.13
Netherlands
14
2.18
1.97
2.14
1.94
1.94
1.61
Finland
15
2.83
3.14
2.36
2.00
1.85
1.51
Spain
16
0.92
1.24
1.36
1.33
1.41
1.24
Belgium
17
1.42
1.50
1.52
1.43
1.35
1.18
Czech Republic
18
1.08
1.00
1.39
1.27
1.47
1.12
Brazil
19
1.14
2.38
2.08
1.78
1.44
1.03
Austria
20
1.24
1.23
1.23
1.08
1.02
0.99
Sweden
21
1.37
1.88
1.22
1.25
1.32
0.94
India
22
1.64
0.79
0.91
0.94
0.91
0.93
Hungary
23
1.19
1.11
1.57
1.37
1.09
0.88
Switzerland
24
0.80
0.89
1.17
1.05
0.97
0.86
Vietnam
25
0.11
0.18
0.41
0.49
0.56
0.71
Slovakia
26
0.31
0.51
1.08
1.09
0.97
0.66
Denmark
27
1.02
0.93
0.82
0.74
0.67
0.63
Canada
28
0.57
0.52
0.72
0.65
0.60
0.61
Taiwan
29
0.26
0.50
0.55
0.67
0.67
0.60
Norway
30
0.46
0.76
0.67
0.62
0.62
0.56




\popis{Table A9 Ukraine: Imports from top thirty partners}



2000
2005
2009
2010
2011
2012*








Total imports, cif, EUR mn
15097.7
28985.3
32571.0
45763.8
59340.2
65867.2
Shares in \% (ranking in 2012)






Russia
1
41.74
35.54
29.13
36.54
35.27
32.39
China
2
0.94
5.01
6.02
2.03
7.59
9.33
Germany
3
8.13
9.36
8.48
7.58
8.31
8.04
Belarus
4
4.31
2.60
3.73
4.23
5.10
5.99
Poland
5
2.24
3.89
4.78
4.59
3.85
4.21
United States
6
2.58
1.96
2.83
2.91
3.14
3.43
Italy
7
2.48
2.85
2.51
2.29
2.43
2.64
Turkey
8
1.15
1.68
2.10
2.14
1.79
2.31
France
9
1.69
2.21
2.14
1.82
1.82
1.97
Korea Republic
10
0.79
1.79
1.25
0.46
1.50
1.83
Kazakhstan
11
2.96
0.52
4.48
1.26
2.03
1.77
Czech Republic
12
1.17
1.64
1.37
1.23
1.43
1.47
Japan
13
0.71
1.52
1.14
1.32
1.23
1.41
Hungary
14
1.19
1.79
1.49
2.00
1.61
1.37
United Kingdom
15
1.45
1.39
1.43
1.35
1.37
1.36
Netherlands
16
1.05
1.28
1.49
1.38
1.44
1.33
India
17
0.54
0.89
1.05
0.28
0.98
1.21
Romania
18
0.35
0.59
1.07
1.12
1.36
1.10
Lithuania
19
0.97
0.55
0.90
1.05
1.00
1.08
Singapore
20
0.03
0.05
0.06
0.01
0.05
0.97
Switzerland
21
1.55
0.70
0.96
0.84
0.96
0.90
Spain
22
0.72
0.65
0.82
0.77
0.83
0.88
Austria
23
1.33
1.27
1.35
1.15
0.86
0.87
Belgium
24
0.97
0.87
1.02
0.97
0.80
0.84
Slovakia
25
0.89
0.84
0.67
0.73
0.73
0.69
Brazil
26
0.67
0.86
0.83
0.17
0.66
0.68
Sweden
27
1.08
1.51
0.99
0.59
0.77
0.64
Finland
28
0.69
0.97
0.93
0.71
0.63
0.57
Indonesia
29
0.20
0.34
0.57
0.20
0.64
0.49
Norway
30
0.32
0.35
0.57
0.43
0.33
0.45




\popis{Table A10 Kazakhstan: Exports and imports by SITC commodity groups}


2000
2005
2009
2010
2011
2012*
Exports 1)






Total exports, fob, EUR mn
9319.0
22370.9
30977.2
45387.1
62928.6
67248.6
Shares in \% 






0  Food and live animals
6.7
2.2
3.5
3.1
1.8
2.9
1  Beverages and tobacco
0.2
0.2
0.1
0.1
0.1
0.1
2  Crude materials, inedible, except fuels
7.5
6.7
6.0
5.4
6.9
6.2
3  Mineral fuels, lubricants and related materials
52.8
70.1
69.5
71.7
72.0
69.9
4  Animal and vegetable oils, fats and waxes
0.0
0.0
0.1
0.1
0.0
0.1
5  Chemicals and related products, n.e.s.
1.1
1.9
4.5
4.4
3.3
3.8
6  Manufactured goods classified chiefly by material
26.9
16.7
13.7
13.0
13.7
14.0
7  Machinery and transport equipment
2.2
1.2
0.9
0.6
0.8
1.4
8  Miscellaneous manufactured articles
0.5
0.2
0.1
0.1
0.3
0.7
9  Commodities not classified elsewhere in the SITC
2.0
0.7
1.5
1.5
1.1
1.0







Imports 1)






Total imports, cif, EUR mn
5329.9
13939.0
20372.8
23440.1
26618.5
36021.2
Shares in \% 






0  Food and live animals
7.1
5.7
6.8
8.0
8.7
7.8
1  Beverages and tobacco
1.3
1.1
1.2
1.0
1.1
1.1
2  Crude materials, inedible, except fuels
2.8
2.0
1.2
1.3
1.4
2.3
3  Mineral fuels, lubricants and related materials
11.4
11.9
10.0
9.9
12.8
10.8
4  Animal and vegetable oils, fats and waxes
0.8
0.4
0.5
0.5
0.5
0.4
5  Chemicals and related products, n.e.s.
10.2
9.3
10.0
11.9
10.4
10.3
6  Manufactured goods classified chiefly by material
18.8
21.6
26.5
18.1
17.3
19.7
7  Machinery and transport equipment
39.7
41.5
37.0
40.3
35.8
38.0
8  Miscellaneous manufactured articles
6.4
6.5
6.8
9.0
11.8
9.4
9  Commodities not classified elsewhere in the SITC
1.4
0.1
0.2
0.1
0.2
0.4
1) Officially registered trade.




\popis{Table A11 Russia: Exports and imports by SITC commodity groups}


2000
2005
2009
2010
2011
2012*
Exports 






Total exports, fob, EUR mn
111449
193709
216560
299354
371071
408182
Shares in \% 






0  Food and live animals
0.9
1.3
2.5
1.6
1.8
2.5
1  Beverages and tobacco
0.1
0.2
0.3
0.2
0.1
0.2
2  Crude materials, inedible, except fuels
4.5
4.4
3.1
3.1
3.3
2.4
3  Mineral fuels, lubricants and related materials
50.6
61.8
63.0
65.6
67.0
70.3
4  Animal and vegetable oils, fats and waxes
0.1
0.1
0.3
0.1
0.2
0.4
5  Chemicals and related products, n.e.s.
6.0
4.2
4.1
4.0
4.2
4.7
6  Manufactured goods classified chiefly by material
17.8
14.8
12.3
11.2
9.8
9.5
7  Machinery and transport equipment
6.2
4.1
3.6
2.8
2.3
2.7
8  Miscellaneous manufactured articles
2.0
0.8
0.8
0.6
0.4
0.6
9  Commodities not classified elsewhere in the SITC
11.8
8.4
10.1
10.8
10.8
6.6







Imports






Total imports, cif, EUR mn
36613
79190
120136
172579
219576
246447
Shares in \% 






0  Food and live animals
15.6
12.8
13.1
11.6
10.1
10.2
1  Beverages and tobacco
3.3
2.4
1.7
1.5
1.3
1.4
2  Crude materials, inedible, except fuels
7.2
3.7
3.0
2.2
2.1
2.2
3  Mineral fuels, lubricants and related materials
4.1
1.6
1.4
1.2
1.6
1.3
4  Animal and vegetable oils, fats and waxes
1.1
0.8
0.7
0.7
0.6
0.5
5  Chemicals and related products, n.e.s.
11.8
12.7
13.1
12.8
11.7
12.1
6  Manufactured goods classified chiefly by material
13.9
13.0
11.5
11.9
11.6
12.8
7  Machinery and transport equipment
24.5
39.9
37.1
39.0
41.9
31.5
8  Miscellaneous manufactured articles
7.2
7.0
10.2
11.0
10.0
11.3
9  Commodities not classified elsewhere in the SITC
11.2
6.2
8.0
8.2
9.1
16.7




\popis{Table A12 Ukraine: Exports and imports by SITC commodity groups}


2000
2005
2009
2010
2011
2012*
Exports






Total exports, fob, EUR mn
15764.6
27455.0
28457.9
38729.2
49129.8
53536.7
Shares in \% 






0  Food and live animals 1)
6.3
10.3
16.8
12.2
11.7
17.4
1  Beverages and tobacco 
.
.
.
.
.
.
2  Crude materials, inedible, except fuels
12.7
7.2
9.6
10.4
11.0
10.3
3  Mineral fuels, lubricants and related materials
5.5
9.8
5.4
7.1
8.3
5.3
4  Animal and vegetable oils, fats and waxes
1.6
1.7
4.4
5.0
4.8
6.0
5  Chemicals and related products, n.e.s.
9.0
9.0
6.2
6.7
7.9
7.3
6  Manufactured goods classified chiefly by material
45.6
44.1
36.1
37.1
33.3
28.8
7  Machinery and transport equipment
12.3
12.6
16.6
17.3
12.9
14.4
8  Miscellaneous manufactured articles
4.5
3.8
4.0
3.5
3.0
2.8
9  Commodities not classified elsewhere in the SITC
2.4
1.6
0.8
0.7
7.1
7.6







Imports






Total imports, cif, EUR mn
15097.7
28985.3
32571.0
45763.8
59340.2
65867.2
Shares in \% 






0  Food and live animals 1)
5.9
6.5
9.5
8.2
6.3
7.2
1  Beverages and tobacco 
.
.
.
.
.
.
2  Crude materials, inedible, except fuels
5.6
3.9
3.4
3.7
2.8
2.6
3  Mineral fuels, lubricants and related materials
43.0
29.5
32.2
32.3
34.6
30.9
4  Animal and vegetable oils, fats and waxes
0.3
0.5
0.7
0.7
0.5
0.4
5  Chemicals and related products, n.e.s.
8.8
11.7
15.3
14.3
11.9
12.1
6  Manufactured goods classified chiefly by material
12.8
14.6
13.7
14.4
12.5
11.6
7  Machinery and transport equipment
17.5
25.0
18.5
19.6
16.6
19.5
8  Miscellaneous manufactured articles
3.6
5.4
5.8
6.0
3.9
5.1
9  Commodities not classified elsewhere in the SITC
2.6
2.9
0.9
1.0
11.0
10.4
1) Including beverages and tobacco.


\popis{Table B1 Tariffs and imports, Russian Federation}









Part A.1
 
Tariffs and imports: Summary and duty ranges
 
 
 
Summary
 
Total
Ag
Non-Ag
  WTO member since
 
 
  2012
Simple average final bound
 
 
7.8  
11.2  
7.2  
  Binding coverage:

Total
  100
Simple average MFN applied
2012  
10.0  
13.3  
9.4  



Non-Ag
  100
Trade weighted average
2011  
9.9  
16.7  
8.8  
  Ag: Tariff quotas (in \%)


    3.2
Imports in billion US\$
2011  
277.6  
37.4  
240.2  
  Ag: Special safeguards (in \% )
 
  0
 
 
 
 
 
 
 
 
 
 
 
Frequency distribution 
Duty-free
0 <= 5
5 <= 10
10 <= 15
15 <= 25
25 <= 50
50 <= 100
> 100
NAV

Tariff lines and import values (in \%) 
in \%
Agricultural products
 







 
  
Final bound
 
    3.0
   43.3
   21.5
   24.5
    4.2
    0.8
    2.3
      0.3
     22.9
MFN applied
2012  
    8.2
   36.9
    7.8
   30.2
   10.7
    3.8
    2.1
      0.3
     28.2
Imports
2011  
    9.0
   24.8
    7.2
   27.9
   18.6
    8.0
    4.4
      0.0
     54.7
 Non-agricultural products







 
  
Final bound
 
    3.4
   50.0
   30.4
   14.9
    1.2
    0.1
    0.0
        0
      7.0
MFN applied
2012  
   14.2
   34.4
   19.9
   20.6
    8.9
    1.5
    0.3
      0.1
     10.1
Imports
2011  
   32.6
   21.9
   17.8
   12.9
    8.5
    6.0
    0.1
      0.0
      9.6











Part A.2
 
Tariffs and imports by product groups
 
 
 
 
 
Final bound duties
MFN applied duties 
Imports 
Product groups
AVG
Duty-free
Max
Binding
AVG
Duty-free
Max
Share
Duty-free
 
 
in \%
 
in \%
 
in \%
 
in \%
 in \%
Animal products
23.1 
      7.4
       80
  100
23.7  
   14.8
       90
    2.5
      3.6
Dairy products
14.9 
        0
       21
  100
18.4  
      0
       50
    0.8
        0
Fruit, vegetables, plants
8.7 
      0.2
       45
  100
11.7  
    4.6
      134
    4.1
      8.6
Coffee, tea
6.4 
      4.2
       13
  100
9.1  
   20.8
       23
    1.1
     34.1
Cereals \& preparations
10.1 
      1.3
       77
  100
12.9  
    3.5
       77
    0.9
      1.6
Oilseeds, fats \& oils
7.1 
      8.2
       25
  100
8.5  
   10.9
       48
    0.8
     22.1
Sugars and confectionery
12.7 
        0
       48
  100
12.9  
      0
       39
    0.7
        0
Beverages \& tobacco
23.6 
        0
      292
  100
29.2  
    5.2
      292
    1.6
      2.7
Cotton
0.0 
    100.0
        0
  100
0.0  
  100.0
        0
    0.1
    100.0
Other agricultural products
5.3 
        0
       10
  100
5.6  
    7.4
       20
    0.8
      7.0
Fish \& fish products
7.5 
        0
       77
  100
12.4  
    0.4
       77
    0.9
      2.6
Minerals \& metals
8.0 
      0.1
       20
  100
9.9  
    6.4
       90
    9.5
     12.8
Petroleum
5.0 
        0
        5
  100
4.5  
   10.0
        5
    1.1
      1.3
Chemicals
5.2 
      0.4
       10
  100
6.4  
    5.8
       21
   13.8
     13.2
Wood, paper, etc.
7.9 
      5.0
       15
  100
12.8  
    6.1
       30
    3.3
      9.8
Textiles
7.8 
        0
       18
  100
10.9  
    0.6
       37
    2.1
      2.7
Clothing
11.8 
        0
       42
  100
19.6  
      0
      100
    2.4
        0
Leather, footwear, etc.
6.4 
        0
       15
  100
10.3  
    8.7
      176
    3.2
      7.7
Non-electrical machinery
5.8 
      7.9
       15
  100
3.4  
   66.2
       21
   18.7
     73.6
Electrical machinery
6.2 
     23.3
       16
  100
7.3  
   25.2
       27
   11.1
     37.8
Transport equipment
8.9 
      2.5
       20
  100
10.6  
   17.7
       35
   16.1
     29.9
Manufactures, n.e.s.
8.4 
      7.9
       20
  100
11.4  
   17.2
      190
    4.4
     39.7
Part B
 
Exports to major trading partners and duties faced
 
 
 
Major markets
Bilateral imports
Diversification
MFN AVG of
Pref.
Duty-free imports

 
in million
95\% trade in no. of
traded TL
margin
TL
Value

 
  US\$
HS 2-digit
HS 6-digit
Simple
Weighted
Weighted
in \%
in \%
Agricultural products
 

 

 
  
  
   


1. Kazakhstan                           
 2011  
1,569 
22  
120  
      20.8
      24.5
      24.5
100.0  
100.0  
2. European Union                       
 2011  
1,510 
22  
55  
      14.6
       7.9
       1.3
21.8  
67.7  
3. Egypt                                
 2011  
1,386 
2  
2  
     112.0
       0.5
       0.0
25.0  
97.4  
4. Turkey                               
 2011  
863 
6  
13  
      29.5
      85.4
       0.0
14.2  
3.7  
5. Ukraine                              
 2011  
679 
12  
49  
       9.8
      10.4
       8.8
96.1  
80.8  
Non-agricultural products

 

 
  
  
   


1. European Union                       
 2011  
241,503 
16  
62  
       4.1
       0.3
       0.2
69.5  
97.1  
2. China                                
 2011  
40,298 
18  
46  
       7.7
       1.4
       0.0
17.1  
73.3  
3. United States                        
 2011  
33,383 
19  
49  
       2.3
       0.2
       0.1
87.7  
33.4  
4. Ukraine                              
 2011  
28,386 
45  
347  
       3.7
       0.9
       0.9
100.0  
100.0  
5. Belarus                              
 2011  
23,958 
50  
540  
       9.5
       3.3
       3.3
100.0  
100.0  
Source: WTO (http://stat.wto.org/TariffProfiles/).



\popis{Table B2 Tariffs and imports, Ukraine}











Part A.1
 
Tariffs and imports: Summary and duty ranges
 
 
 
Summary
 
Total
Ag
Non-Ag
  WTO member since
 
 
  2008
Simple average final bound
 
 
5.8  
11.0  
5.0  
  Binding coverage:

Total
  100
Simple average MFN applied
 
2012  
4.5  
9.5  
3.7  



Non-Ag
  100
Trade weighted average
 
2011  
2.7  
9.1  
2.2  
  Ag: Tariff quotas (in \%)

    0.1
Imports in billion US\$
 
2011  
82.2  
5.8  
76.3  
  Ag: Special safeguards (in \% )
 
  0
 
 
 
 
 
 
 
 
 
 
 
Frequency distribution 
Duty-free
0 <= 5
5 <= 10
10 <= 15
15 <= 25
25 <= 50
50 <= 100
> 100
NAV

Tariff lines and import values (in \%) 
in \%
Agricultural products
 







 
  
Final bound
 
   12.6
   19.6
   27.5
   13.9
   25.5
    0.8
    0.1
        0
      1.0
MFN applied
2012  
   21.1
   22.0
   26.3
   12.1
   17.5
    0.8
    0.2
      0.1
        0
Imports
2011  
   39.3
   20.4
   28.4
    5.8
    1.6
    3.8
    0.7
      0.0
        0
 Non-agricultural products
 







 
  
Final bound
 
   33.8
   16.8
   43.0
    5.8
    0.5
      0
      0
        0
      0.0
MFN applied
2012  
   43.1
   29.9
   21.3
    5.4
    0.3
      0
    0.0
        0
        0
Imports
2011  
   66.4
   18.1
   14.7
    0.9
    0.1
      0
      0
        0
        0











Part A.2
 
Tariffs and imports by product groups
 
 
 
 
 
Final bound duties
MFN applied duties 
Imports 
Product groups
AVG
Duty-free
Max
Binding
AVG
Duty-free
Max
Share
Duty-free
 
 
in \%
 
in \%
 
in \%
 
in \%
 in \%
Animal products
13.0 
        0
       20
  100
11.0  
    9.0
       20
    0.5
     15.0
Dairy products
10.0 
        0
       10
  100
10.0  
      0
       10
    0.2
        0
Fruit, vegetables, plants
13.1 
     10.2
       20
  100
9.9  
   18.9
       20
    1.4
     54.6
Coffee, tea
5.8 
     35.4
       20
  100
5.8  
   35.4
       20
    1.3
     42.0
Cereals \& preparations
12.7 
      3.3
       20
  100
12.6  
    3.8
       20
    0.9
     27.1
Oilseeds, fats \& oils
10.7 
     11.0
       30
  100
8.3  
   20.1
       30
    0.9
     89.9
Sugars and confectionery
17.5 
      0.6
       50
  100
17.5  
      0
       50
    0.3
        0
Beverages \& tobacco
7.9 
     25.7
       64
  100
12.2  
   26.2
      424
    1.2
     23.9
Cotton
1.4 
     40.0
        5
  100
1.4  
   40.0
        5
    0.0
     61.3
Other agricultural products
7.6 
     23.9
       20
  100
5.5  
   45.2
       20
    0.5
     19.3
Fish \& fish products
3.7 
     61.7
       20
  100
2.6  
   68.2
       20
    0.7
     68.0
Minerals \& metals
4.5 
     42.4
       20
  100
3.0  
   47.6
       20
   32.8
     79.0
Petroleum
1.5 
     72.0
       10
  100
0.9  
   84.3
       10
   13.7
     97.3
Chemicals
5.1 
     16.1
       10
  100
3.2  
   39.3
       65
   12.7
     55.4
Wood, paper, etc.
0.4 
     95.8
       10
  100
0.3  
   95.8
       10
    3.1
     99.1
Textiles
4.1 
     33.7
       13
  100
3.8  
   35.6
       13
    2.1
     25.6
Clothing
11.4 
      1.0
       12
  100
11.3  
    1.1
       12
    0.6
      0.1
Leather, footwear, etc.
7.2 
     14.9
       25
  100
5.4  
   27.0
       25
    1.9
     20.3
Non-electrical machinery
4.2 
     38.7
       12
  100
2.1  
   51.3
       10
    8.8
     62.4
Electrical machinery
5.3 
     33.0
       25
  100
3.8  
   39.1
       25
    6.8
     34.2
Transport equipment
7.5 
     15.8
       20
  100
5.1  
   39.6
       20
    7.5
     21.4
Manufactures, n.e.s.
6.4 
     31.9
       25
  100
5.5  
   32.0
       25
    2.1
     68.6
Part B
 
Exports to major trading partners and duties faced
 
 
Major markets
Bilateral imports
Diversification
MFN AVG of
Pref.
Duty-free imports

 
in million
95\% trade in no. of
traded TL
margin
TL
Value

 
  US\$
HS 2-digit
HS 6-digit
Simple
Weighted
Weighted
in \%
in \%
Agricultural products
 

 

 
  
  
   


1. European Union                       
 2011  
3,627 
15  
25  
      13.7
       4.5
       0.9
27.3  
70.6  
2. Russian Federation                   
 2011  
2,093 
17  
77  
      15.6
      20.9
      20.9
99.7  
100.0  
3. Turkey                               
 
 2011  
1,183 
4  
11  
      30.6
      46.7
       0.0
12.7  
2.2  
4. Egypt                                
 
 2011  
997 
3  
5  
       4.6
       0.2
       0.0
24.3  
95.3  
5. India                                
 
 2011  
903 
1  
1  
      40.2
       2.2
       0.0
15.6  
95.9  
Non-agricultural products
 

 

 
  
  
   


1. Russian Federation                   
 2011  
17,846 
45  
514  
       9.6
       7.9
       7.9
100.0  
100.0  
2. European Union                       
 2011  
14,866 
37  
249  
       4.0
       0.9
       0.6
71.5  
90.8  
3. Turkey                               
 
 2011  
3,564 
14  
50  
       5.0
       7.3
       0.7
66.1  
39.6  
4. China                                
 
 2011  
3,174 
11  
17  
       8.0
       0.9
       0.0
16.1  
77.7  
5. Belarus                              
 
 2011  
1,615 
43  
477  
       9.4
       7.3
       7.3
100.0  
100.0  
\zdroj{Source: WTO (http://stat.wto.org/TariffProfiles/).}
