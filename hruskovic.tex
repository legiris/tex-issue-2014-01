\newpage\nuluj
\pagestyle{hruskovic}

\logo

\nadpis{Legal Aspects of the Association of Overseas\\Countries and Territories with the European Union}

\autor{Ivan Hru�kovi�\pozn{Comenius University in Bratislava, Faculty of Law, Department of International Law and European Law, P.\;O. Box 313, 810\;00 Bratislava 1, Slovakia. E-mail: ivan.hruskovic@flaw.uniba.sk; ivanhruskovic@chello.sk.}}
\linka{3ex}

\cast{Abstract}
This article is concerned with an issue in European law that is historically derived from the Treaty establishing the European Economic Community. The aim is to illuminate the functions and objectives of the European Union that correspond to the provisions of Part Four of the Treaty on the Functioning of the European Union -- Association of the Overseas Countries and Territories. The basic provisions are laid down in the primary law of the European Union but there are also other documents and legislative acts that lay down more detailed provisions on the given issue and reveal a more complex perspective. The article focuses on the existing legislation regulating the association of overseas countries and territories with the Union. The author first draws attention to the key provisions of the Treaty establishing the EEC that define the purpose, objectives and fundamental principles applied between the Union and the overseas countries and territories. The main purpose of the article, however, is to analyse the current legal basis of association and to highlight the need for a new legislative framework of cooperation. New legislation should take into account not only the interests of the Union, but also the desire of the overseas countries and territories for a new quality of mutual cooperation. The author argues that the best way to improve the association mechanism based on the Lisbon Treaty is to modernise the Union's existing secondary legislation in this area. In this regard the author analyses issues related to Council decision 2001/822/EC on the association of the overseas countries and territories with the European Community and the proposal for a Council decision (COM/2012/362) of 16 July, 2012.\\

\cast{Keywords}
European Union, European Law, Primary Law, Council Decisions, Association of Overseas Countries and Territories\hrule height 0pt depth 3pt width 0pt
\linka{4ex}
    
\podnadpis{Introduction}
The current legal framework for the association of overseas countries and territories (hereinafter ``OCTs'') with the Union builds on the provisions on support for economic and social development in the original Treaty establishing the European Economic Community (hereinafter the ``EEC'') of 1957 (articles 131--136). The development of economic and later social and cultural relations of the Member States of the Community and the Community as a whole with ``non-European'' countries and territories is characterised as ``special relations'' that were historically determined and linked initially with the countries Belgium, France, Italy and the Netherlands. Article 131 of the Treaty establishing the EEC lists these Member States and refers to a list of the relevant countries and territories in Annex IV of the Treaty.\pozn{Annex IV. Overseas countries and territories to which the provisions of Part Four of the Treaty apply: French West Africa (Senegal, French Sudan, French Guinea, Ivory Coast, Dahomey, Mauritania, Niger, and Upper Volta), French Equatorial Africa (Middle Congo, Ubangi-Shari, Chad and Gabon), Saint Pierre and Miquelon, the Comoro Archipelago, Madagascar and dependencies, French Somaliland, New Caledonia and dependencies, French Settlements in Oceania, Southern and Antarctic Territories, the Autonomous Republic of Togoland, the Trust Territory of the Cameroons under French Administration, the Belgian Congo and Ruanda-Urundi, the Trust Territory of Somaliland under Italian Administration, Netherlands New Guinea.\\
See Annex IV. Overseas countries and territories, Treaty establishing the EEC (Euroskop, 2013).}

The basis for equal relations between overseas countries and territories (non-European countries and territories) on the one side and the Community and its Member States on the other is therefore the Treaty establishing the EEC. The fundamental provision for legislation on association and the provision that defines its main purpose is the second paragraph of Article 131 of the Treaty establishing the EEC: ``\dots\; to establish close economic relations between them and the Community as a whole.''\pozn{Quoted from Article 131 of the Treaty establishing the EEC (Euroskop, 2013, p.\;51).} Article 132 of the Treaty establishing the EEC then gives a more detailed presentation of the objectives of association. Article 132 also presents a suitable basis for the advantages that Member States of the Community and the Community as a whole provide in certain areas for non-European countries and territories. It lays down the principle of equal treatment for non-European countries and territories in commercial exchanges, not only between them and the European states with which they have a special relationship, but also with the other Member States of the Community. It permits natural persons and legal entities (nationals of Member States of the Community and non-European countries and territories) to participate on equal terms in tenders and supplies for investments financed by the Community. It includes provisions that create preconditions for the right of settlement on a non-discriminatory basis. This legislation provides a good basis for equal relations between these countries and territories and the Community and they are still reflected in current law based on the Treaty on the functioning of the European Union (hereinafter the ``TFEU''). It is understandable that the provisions of the Treaty establishing the EEC should open the path to the principle of equal treatment between EEC Member States and non-European countries and territories, particularly in economic and commercial cooperation. The above articles of the Treaty establishing the EEC are elaborated on by articles 133--136 (including the principle of equal treatment), which lay down:
\zacseznam
\item a path to the abolition of customs duties on goods imported from non-European countries and territories (the gradual abolition of customs duties on goods imported from Member States and non-European countries and territories in accordance with the corresponding articles of the Treaty establishing the EEC),
\item an important basis for subsequent law on the free movement of workers from non-European countries and workers from Member States.
\item the setting of future legal procedure and particulars of the association of non-European countries and territories with the Community by means of implementing conventions,
\item the Council's competence to decide on the provision to be made for a further period based on the results achieved and on the principles set out in the Treaty.
\konseznam
The legislative provisions in the Treaty establishing the EEC laid a foundation for cooperation in economic and commercial matters. Association gradually developed to include issues connected with social development and the potential development of the work force. The Treaty foresaw that in the future there would of necessity be a positive influence and benefits from cooperation in the area of further social development and cultural and regional cooperation. The aim of the following parts of the article is to highlight certain aspects of current legislation on the association of overseas countries and territories with the Union and to analyse the rules, objectives and principles that apply to relations between them. This information is used to analyse the need for a new legal framework based on a~new quality in the implementation of common interests, reciprocity, rights and obligations.

\podnadpis{Legal basis of the Association of Overseas Countries and Territories with the Union}
In every revision of the foundational treaties or primary law, the treatment of this issue was based on the provisions of the original Treaty establishing the EEC. The current form of the legislation, laid down by the Lisbon Treaty in articles 198--204 of the TFEU, corresponds in many ways to the formulation in the original primary law. It also builds on the basic purpose of association defined in paragraph 2 of article 198 of the TFEU, which is ``to promote the economic and social development of the countries and territories and to establish close economic relations between them and the Union as a whole''. It defines the basic mechanism of cooperation and partnership in association as a trilateral partnership involving the Commission, the Member State with which an OCT has special relations and the country or territory itself.\pozn{For further information on trilateral partnership and cooperation under association see Syllov�, P�trov�, Paldusov� (2010, p.\;714).} Association is intended primarily to further the interests and prosperity of the inhabitants of the OCTs in order to lead them to successful economic, social and cultural development. At the same time, a basic prerequisite and starting point for association arrangements is that they must be in accordance with the principles representing the fundamental values of the Union. According to Article 198 of the TFEU ``the Member States agree to associate with the Union the non-European countries and territories which have special relations with Denmark, France, the Netherlands and the United Kingdom''. A list of the relevant OCTs is given in Annex II of the TFEU ``Overseas countries and territories to which the provisions of Part Four of the Treaty on the functioning of the European Union apply''.\pozn{Greenland, New Caledonia and Dependencies, French Polynesia, French Southern and Antarctic Territories, Wallis and Futuna Islands, Mayotte, Saint Pierre and Miquelon, Aruba, Netherlands Antilles: Bonaire, Cura�ao, Saba, Sint Eustatius, Sint Maarten, Anguilla, Cayman Islands, Falkland Islands, South Georgia and the South Sandwich Islands, Montserrat, Pitcairn, Saint Helena and Dependencies, British Antarctic Territory, British Indian Ocean Territory, Turks and Caicos Islands, British Virgin Islands, Bermuda (Siman and Sla��an, 2010, p.\;898).\\
Article 204 of the TFEU regulates the application of Articles 198--203 of the TFEU in relation to Greenland. It recognises an exception for Greenland from the association regime although in general Part Four applies to Greenland. The basic legislation for this is Protocol 34 on Special Arrangements for Greenland annexed to the Treaty. The regime is further regulated by Commission notices and regulations, Council regulations and decisions and agreements between the European Communities (the Union), Greenland and the Member States (for example, EC -- the government of Denmark and Greenland). In 2011, a proposal was made for new legislation in the form of a proposal for a Council decision on relations between the European Union on the one hand and Greenland and the Kingdom of Denmark on the other (COM/2011/846).} Like the original Treaty establishing the EEC\pozn{Article 136(2) of the Treaty establishing the EEC: ``Before the expiry of the Convention \dots the Council, acting by means of a unanimous vote, shall, proceeding from the results achieved and on the basis of the principles set out in this Treaty, determine the provisions to be made for a further period''.\\
Quoted from the second paragraph of Article 136 of the Treaty establishing the EEC (Euroskop, 2013, p.\;52).} and the later revisions of the foundational treaties, the TFEU gives the main voice in further legislation on issues relating to association with OCTs to the bodies of the Union. In addition to the traditional references to the Council and the Commission, the primary law also brings in the European Parliament. The powers of the above bodies are set out in article 203 of the TFEU, which recognises the right of the Council and the Commission to lay down detailed rules and the procedure for association. The relationship between the two bodies is that the Council lays down the rules and the procedure unanimously based on a proposal from the Commission. At the same time, the bodies of the Union must act on the basis of the experience acquired under association and of the principles set out in the Treaties.\pozn{Refers to the general principles (Article 2 of the Treaty on European Union refers to values) on which the EU is founded. These are defined in the Treaty on European Union (hereinafter the ``TEU'') and the TFEU, and according to the judgement of the Court of Justice in case C-17/98 Emesa Sugar (Free Zone) NV v. Aruba of 8\;February, 2000, account must be taken both of the principles set out in Part Four of the Treaty and of the other principles of Community law, including those relating to the common agricultural policy (previously Article 38 of the Treaty establishing the European Community, not Article 38 of the TFEU).\\
For further information on the principles enshrined in Article 2 of the TEU, Article 1 and following of the TFEU and Article 38 of the TFEU, see the work cited in note no. 5, p.\;758, 782 and 791.\\
For further information on case C-17/98 Emesa Sugar (Free Zone) NV v. Aruba, see Judgement of the Court of 8 February, 2000 (Court of Justice, 2000).}

The basic legal framework for association of OCTs is developed in more detail by Council Decision 2001/822/EC of 27 November, 2001, on the association of the overseas countries and territories with the European Community,\pozn{The content structure of Council Decision 2001/822/EC has a comprehensive character and the provisions concerning the issue of overseas association are specified in detail. The decision specifies the reason for its adoption with reference to the Treaty establishing the European Community (in particular Article 187) and the proposal of the Commission. The decision has four main parts, four annexes and seven appendices to the annexes.\\
Content structure of the decision:\\
Part One: General provisions of the association of the OCTs with the community (Articles 1--9)
\zacseznam
\item Chapter 1: General provisions
\item Chapter 2: Actors of cooperation in the OCTs
\item Chapter 3: Principles and Procedures of the OCT-EC Partnership
\konseznam
Part Two: The Areas of OCT -- EC Cooperation (Articles 10--17)\\
Part Three: Instruments of OCT -- EC Cooperation (Articles 18--60)\\
\hspace*{2ex}Title I. Development Finance Cooperation
\zacseznam
\item Chapter 1: General provisions
\item Chapter 2: Resources made available to the OCTs
\item Chapter 3: Private sector investment support
\item Chapter 4: Additional support in the event of fluctuations in export earnings
\item Chapter 5: Support for other actors of cooperation
\item Chapter 6: Support for humanitarian and emergency aid
\item Chapter 7: Implementation procedures
\item Chapter 8: Transition from previous European Development Funds (EDFs) to the 9th EDF 
\konseznam
\hspace*{2ex}Title II. Economic and trade cooperation
\zacseznam
\item Chapter 1: Arrangements for trade in goods
\item Chapter 2: Trade in services and rules of establishment
\item Chapter 3: Trade-related areas
\item Chapter 4: Monetary and tax matters
\item Chapter 5: Vocational training, eligibility for Community programmes and other provisions
\konseznam
Part Four, Final provisions (Articles 61--64).} Article 2 of which states that association shall be based on the principles of liberty, democracy, respect for human rights and fundamental freedoms and the rule of law These principles shall be common to the Member States and the OCTs linked to them. There shall be no discrimination based on sex, racial or ethnic origin, religion or belief, disability, age or sexual orientation in the areas of cooperation referred to in this Decision.

The amendment of the foundational treaties by the Lisbon Treaty also affected the question of the adoption of rules and the procedure for association, to which it added the possibility of adoption in accordance with a special legislative procedure. This extended the previous provisions set out in Article 187 of the Treaty establishing the EC to the effect that where the provisions in question are adopted by the Council in accordance with a special legislative procedure, it shall act unanimously on a proposal from the Commission and after consulting the European Parliament. Special legislative procedure is defined in Article 289(1) of the TFEU. The use of a special legislative procedure to define detailed rules and procedures for association (adoption by the Council on a proposal from the Commission and after consulting the European Parliament) is possible when the Treaties require it, or ``in the specific cases provided for by the Treaties''.

The status of associated OCTs can change based on measures that the Council is competent to adopt having regard for the objectives laid down in the foundational treaties (the curtailment or cancellation of certain advantages). On the other hand, the list of OCTs in Annex II of the TFEU can also be enlarged, as in the case of the Council Decision of 29 October, 2010, amending the TFEU, by which the French Collectivity of Saint Barth�lemy became an overseas country and territory associated with the Union. This involved an amendment of the TFEU, and in particular Annex II thereto, by including the island of Saint-Barth�lemy in the list of OCTs to which the fourth part of the Treaty applies.\pozn{On the inclusion of Saint-Barth�lemy in the list in Annex II of the TFEU, see COM/2012/061 final-2012/0024 (CNS) (Council, 2012).}

Articles 199--203 of the TFEU define the conditions of association in the form of objectives and principles,\pozn{P.\;Svoboda notes that the formulation or presentation of Article 199 of the TFEU is not precise or the principles of association are erroneously referred to as ``objectives'' of association (Svoboda, 2010, p.\;2005).} which shall be applied in mutual relations between Member States and OCTs.
\zacseznam
\item Member States shall apply to their trade with OCTs the same treatment as they accord each other. Likewise, each OCT shall apply to its trade with Member States the same treatment as that which it applies to the European State with which it has special relations.
\item The Member States shall contribute to the investments required for the progressive development of OCTs. For investments financed by the Union, participation in tenders and supplies shall be open on equal terms to all natural and legal persons who are nationals of a Member State or of one of the countries and territories.
\item In relations between Member States and OCTs, the right of establishment of nationals and companies or firms shall be regulated in accordance with the provisions and procedures laid down in the Chapter of the TFEU relating to the right of establishment (Articles 49--55 of the TFEU). The provisions guarantee equal conditions for establishment not only from the side of the Member States but also oblige the OCT authorities to afford equal treatment for nationals and companies or enterprises from Member States of the Union in relation to the right of establishment, with treatment no less favourable than that which they extend to subjects from third countries. The OCT authorities shall not discriminate between nationals, companies or enterprises of Member States. These issues are thus subject to a prohibition of discrimination, which can be derogated from (an exception can be made) by a~procedure pursuant to Article 203 of the TFEU.\pozn{Article 203 of the TFEU and Article 44 of Council Decision 2001/822/EC permits an OCT to adopt regulations concerning, for example, restrictions on the movement of workers if this promotes or supports local employment. In this case, the OCT authorities shall notify the Commission of the regulations they adopt so that it may inform the Member States. For further information, see Council decision 2001/822/EC (Council, 2001, p.\;31--32).}
\item Customs duties on imports into the Member States of goods originating in OCTs shall be prohibited in conformity with the prohibition of customs duties between Member States in accordance with the provisions of primary law (prohibition of customs duties and charges having equivalent effect and also the prohibition of customs duties of a fiscal nature). Customs duties on imports into each OCT from Member States or from the other OCTs are also prohibited in accordance with the provisions of primary law. On the other hand, there is an exception whereby in certain cases an OCT can levy customs duties (provided that they meet the OCT's development and industrialisation needs) or customs duties of a fiscal nature which are part of the budget of the OCT. These customs duties must, however, not exceed the level of those imposed on imports of products from the Member State with which the OCT has special relations,\pozn{Article 40 of Council decision 2001/822/EC permits the OCT authorities to retain or introduce, in respect of imports of products originating in Member States, such customs duties or quantitative restrictions as they consider necessary in view of present development needs.}
\item If the level of the duties applicable to goods from a third country on entry into an OCT is liable (when the above principles concerning customs duties have been applied) to cause deflections of trade to the detriment of any Member State, the latter may request the Commission to propose to the other Member States the measures needed to remedy the situation. This permits the situation described above to be rectified through the Commission,
\item Subject to provisions relating to public health, public security or public policy, freedom of movement within Member States for workers from OCTs, and within the OCTs for workers from Member States, shall be regulated by acts adopted in accordance with Article 203 of the TFEU (for Article 203 of the TFEU, see above). There is, however, a requirement that the adopted acts must not interfere with general principles for the protection of public health, public security and public policy.
\konseznam

\podnadpis{The need for a new legislative framework and its basis}
Council decision 2001/822/EC on the association of the OCTs with the EC is (after the primary law) the basis for the regulation of relations in the association of the OCTs. It is, however, applicable only until 31 December, 2013. In addition, new political priorities have emerged at a European and international level, including the need to deal with environmental problems, climate change, and changes in global trade patterns. Since the adoption of the Council decision, the regional and international environments in which OCTs operate have also changed considerably. For these reasons the stakeholders (the Commission, the OCTs, the Member States to which the OCTs are linked and other stakeholders) began work on specific legislative proposals intended to lead to a new document putting relations with OCTs on a new basis and replacing Council decision 2001/822/EC (which has been in effect since 2001).

Preparation of the new legislation began with public consultation on the green paper on future relations between the EU and OCTs (green paper adopted on 25 June, 2008, COM/2008/383). The result of consultation on the green paper was the adoption of a~consensus of the stakeholders on a number of general issues leading to a declaration that current measures in the relations between the OCTs and the EU no longer reflect reality and should be replaced by a new approach.

The resulting Communication from the Commission to the European Parliament, the Council, the European Economic and Social Committee and the Committee of the Regions -- Elements for a new partnership between the EU and the overseas countries and territories of 6\;November, 2009 (COM/2009/623) states that the special relationship between the EU and the OCTs should move away from a classic development cooperation approach to a~reciprocal partnership to support the OCTs' sustainable development and promote the EU's values and standards in the wider world. The principles of association set out in the primary law and the Council decision of 2001 should be replaced by a more contemporary approach. Solidarity implies that the EU should promote the OCTs' sustainable development, in its economic, social and environmental dimensions. Future relations and partnership should be more reciprocal, based on give and take (a relationship between the OCTs and the EU based on mutual interests, reciprocity, rights and obligations). They must also take due account of the OCTs' specificities, in particular their economic and social development, diversity and vulnerability, as well as their environmental importance. The Commission communication (COM/2009/623) proposed that the new partnership between the EU and the OCTs should have three central objectives tailored to the OCTs' specificity: 1. enhancing competitiveness; 2. strengthening resilience and 3. promoting cooperation.\pozn{For further information on the Commission communication (COM/2009/623), see Eur-Lex-52009DC0623-SK (European Commission, 2009).}

In 2012, the Commission presented the results of its work to develop new legislation regulating the association of OCTs with the EU. This is the Proposal for a Council decision on the association of the overseas countries and territories with the European Union (``Overseas Association Decision'') of 16 July, 2012 (COM/2012/362). The Proposal for a decision was preceded by a Commission staff working document accompanying the Council decision on the association of the overseas countries and territories with the European Union (SWD/2012/193).\pozn{Full title of document: Commission staff working document. Executive summary of the impact assessment. Document accompanying the Council decision on the association of the overseas countries and territories with the European Union (SWD/2012/193).} The Commission document takes over the objectives of association developed in the Commission communication (COM/2009/623) when it states that the purpose and objectives and/or principles of association known from primary law (Articles 198--199 of the TFEU) would need to be translated into the objectives identified by the Commission in the communication (COM/2009/623) and which were endorsed by the Council. According to the Commission, the specific objectives of the next association framework would be as follows:
\zacseznam
\item to help promote the EU's values and standards in the wider world,
\item to establish a more reciprocal relationship between the EU and OCTs based on mutual interests,
\item to enhance OCTs' competitiveness,
\item to strengthen OCTs' resilience, reduce their economic and environmental vulnerabilities,
\item to promote cooperation of OCTs with third partners,
\item to integrate the latest EU policy agenda priorities,
\item to take into account changes in global trade patterns and EU trade agreements with third partners.\pozn{For further information, see SWD/2012/193 (European Commission, 2012a, p.\;5).}
\konseznam
The Commission also commented on the possibility of implementing policy to amend the legislation governing OCT association with the EU based on previous consultation, analyses and proposals. The Commission's opinion was based on public consultations, regular ad hoc meetings of the OCTs, the Member States to which the OCTs are linked and the Commission, and forms of dialogue such as annual forums, regular trilateral meetings and partnership working parties dedicated to environmental issues, trade issues, regional integration of the OCTs, financial services in the OCTs and the future EU-OCT relations. In the Commission staff working document (SWD/2012/193), it was proposed that the preferred option should be the modernisation of the Overseas Association Decision and alignment with EU policy framework, since this would best reflect:
\zacseznam
\item the shared ambition of the European Commission, the OCTs, the Member States to which the OCTs are linked, and the EU to review and revise the EU-OCT association, and to establish a more reciprocal partnership, based on mutual interests and taking into account the various challenges OCTs face,
\item the purpose and general objectives of the EU-OCT association as set out in Part Four of the TFEU,
\item the specific objectives of the next association framework.
\konseznam
The resulting legislation in the form of the abovementioned Council decision on the association of the overseas countries and territories with the European Union (``Overseas Association Decision'') will replace Council decision 2001/822/EC. The proposal for a~Council decision will be in accordance with the provisions of the TFEU and will have the general objective of renewing the association, and focusing its areas of cooperation around priorities recognised by all parties as being of mutual interest. The proposal for a~new Council decision should:
\zacseznam
\item set the legal framework,
\item define the General Framework of the EU-OCT Association,
\item identify the possible areas of cooperation between the EU and the OCTs,
\item set out the trade regime that will govern the exchanges and the cooperation in that field between OCTs and the EU,
\item define the different financial instruments to which OCTs will be eligible (11\numend{th} EDF and the horizontal programmes)
\konseznam
The proposal for a Council decision is a legislative proposal that takes into account the political orientations of the Council of the EU and the requests the OCTs and the Member States that have special relations with them have expressed in previous discussions, consultations and proposals in this area. The structure of the proposed new Council decision is qualitatively different from the Decision of 2001 and covers a very broad range of areas.\pozn{The Proposal for a Council Decision (COM/2012/362) has five basic parts: 1. General provisions of the Association of the Overseas Countries and Territories with the Union (Articles 1--13); 2. Areas of cooperation for sustainable development in the framework of the Association (Articles 14--39); 3. Trade and trade-related cooperation (Articles 40--72); 4. Instruments for sustainable development (Article 73--88); 5. Final provisions (Articles 89--94). It has eight annexes and thirteen appendices.} The introductory provisions of the decision are also important for the applications of the principles laid down in the primary law of the EU. Article 1 sets out the purpose of association, which is a partnership based on Article 198 of the TFEU to support the OCTs' sustainable development as well as to promote the values and standards of the Union in the wider world. The partners to the association are the Union, the OCTs and the Member States to which they are linked. In a change from previous legislation, Article 2 specifies the objectives, principles and values of association. Association is based on objectives, principles and values shared by the OCTs, the Member States to which they are linked and the Union. Its function is to pursue the objectives laid down in Article 199 of the TFEU. This will be achieved through the enhancement of the OCT's competitiveness, the strengthening of the OCTs' resilience, the reduction of their vulnerability and the promotion of cooperation between them and other partners. In pursuing those objectives, the association shall respect the fundamental principles of liberty, democracy, human rights and fundamental freedoms, the rule of law, good governance and sustainable development, all of which are common to the OCTs and the Member States to which they are linked. Article 2 prohibits discrimination in the same way as Article 2 of the previous Council decision 2001/822/EC. On the other hand, Article 2(5) recognises partners' rights to determine their own sustainable development policies and priorities, to establish their own levels of domestic environmental and labour protection, and to adopt or modify accordingly the relevant laws and policies, consistently and with commitment to internationally recognised standards and agreements. In exercising their rights, partners shall strive to ensure high levels of environmental and labour protection. Article 2(6) stipulates that implementation of the decision must be guided by the principles of transparency, subsidiarity and the need for efficiency, and equally address the three pillars of OCTs' sustainable development -- economic development, social development and environmental protection. The content of Article 5 on mutual interests, complementarity and priorities is also new, in particular the identification of seven priority areas for cooperation.\pozn{1. The economic diversification of OCT economies, including their further integration in world and regional economies; 2. The promotion of green growth; 3. The sustainable management of natural resources; 4. The adaptation to and mitigation of impacts of climate change; 5. The promotion of disaster risk reduction; 6. The promotion of research, innovation and scientific cooperation activities and 7. The promotion of social, cultural and economic exchanges between the OCTs, their neighbours and other partners.}
    
\podnadpis{Conclusion}
The process of modernising the legislation governing the association of OCTs with the European Union described above has involved a movement away from the traditional understanding of the EU-OCT relationship as that of ``donor and beneficiary''. It responds to the ambition of the OCTS to change the focus of association from poverty reduction and development cooperation to a reciprocal relationship focussed on sustainable development of the OCTs and also taking into account the political orientations of the Union. The aim is to promote a development model that conciliates economic activities and social well-being in the long run while preserving natural resources and ecosystems for future generations

Association with the Union is based on constitutional ties that link OCTs to the Member States listed in Article 198 of the TFEU (Denmark, France, the Netherlands and the United Kingdom). It is accepted that OCTs have wide ranging autonomy, covering areas such as economic affairs, employment market, public health, home affairs and customs. The OCTs are not part of the Union's customs territory and are outside the Internal Market. OCT inhabitants hold EU citizenship because they are nationals of the EU Member States to which their countries and territories are constitutionally linked.\\
    
\cast{References}
Council. (2001). \fntit{Council decision of 27 November, 2001, on the association of the overseas countries and territories with the European Community}. Retrieved from http://www.eur.\\lex.europa.eu/LexUriServ.do?uri=CELEX:32001D0822:CS:HTML.

Council. (2012). \fntit{Council decision on the island of Saint-Barth�lemy (COM/2012/061)}. Retrieved from http://www.eur-lex.europa.eu/LexUriServ/LexUriServ.do?uri=OJ:L:2012:\\264:0001:0002:SK:PDF.

Court of Justice. (2000). \fntit{Case C-17/98 Emesa Sugar (Free Zone) NV v. Aruba}. Retrieved from http://www.curia.europa.eu/juris/liste.jsf?language=ennum=C-17/98.

European Commission. (2012). \fntit{Proposal for a Council decision on the association of the overseas countries and territories with the European Union (COM/2012/362)}. Retrieved from http://www.eur.lex.europa.eu/LexURIServ/LexUriServ.do?uri=COM:2012:0362:\\FIN:SK:PDF. 

European Commission. (2012a). \fntit{Commission staff working document on the association of the overseas countries and territories with the European Union (SWD/2012/193)}. Retrieved from http://www.eu.lex.europa.eu/LexUriServ/LexUriServ.do?uri=SWD:2012:0193:FIN:\\SK:PDF.  

European Commission. (2009). \fntit{The resulting Communication from the Commission to the European Parliament, the Economic and Social Committee and Committee of regions of 6 November, 2009 (COM/2009/623}. Retrieved from EUR-Lex-52009DC0623-Sk. 

Euroskop. (2013). \fntit{Treaty establishing the EEC}. Retrieved from http://www.euroskop.cz/ga\-lery/2/754-smoluva o es.pdf. 

Siman, M., Sla��an, M.  (2010). \fntit{Prim�rne pr�vo Eur�pskej �nie (aplik�cia a v�klad pr�va �nie s judikat�rou)}. Bratislava: Euroiuris.

Svoboda, P. (2010). \fntit{Pr�vo vnej��ch vztahu E� (the Law of EU external relations)}. Prague: C.\;H. Beck.

Syllov�, J., P�trov�, L., Paldusov�, H. (2010). \fntit{Lisabonsk� smlouva. (The Lisbon Treaty) A~commentary}. Prague: C.\;H. Beck.
	
