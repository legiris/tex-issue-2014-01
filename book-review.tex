\newpage\nuluj
\pagestyle{nordhaus}

\autorb{WILLIAM NORDHAUS}
\nadpisb{\raggedright The Climate Casino:\\Risk, Uncertainty, and Economics for a~\mbox{Warming} World}
\vyd{Yale University Press, 2013}{ISBN 978-0300189773}
\uvod{casino-gray.jpg}

Pitched at a general audience, The Climate Casino explains pretty much all the economics necessary for an understanding of climate change policy, along with the related science and politics. As such, it's a vitally important resource, filling a much-needed gap in the vast number of popular books on climate science and climate politics.

The emphasis throughout is on making the key ideas and results as clear as possible, which involves some simplification. So concepts such as integrated assessment models, externalities, discount rates, carbon prices, tipping points and cost-benefit analyses are introduced only when necessary, and are explained in context. Nordhaus also presents broad conclusions without working through the details. (None of the economics involved is controversial, however, and where there is uncertainty or debate he makes that clear.) The result is that The Climate Casino should be quite broadly accessible. It would be a challenge for a reader with no experience of thinking about economics, but not an impossible one. 

In a way, climate change starts with economics and Nordhaus, whose own research is focused on this area, begins by looking at likely future greenhouse emissions under various models for economic growth, carbon intensity and so forth. Significant uncertainty here comes from not being able to predict changes in technology. He goes on to give a~brief summary of climate science, looking at predictions for the next century or so, and introducing the possibility of climate tipping points, fairly abrupt transitions which are irreversible in the short term. 

Next comes a survey of the likely impact of climate change on agriculture, health, oceans, hurricane intensity, and biodiversity. Adding up the damage from these suggests that the total cost will be small relative to expected changes in economic activity over the next century, but there is considerable uncertainty and the impacts are concentrated in unmanaged, non-market areas, where the uncertainties are highest. 

Responding to these impacts will involve adaptation (making changes to economies and societies) and possibly geoengineering, but there are limits to these and mitigation (reducing carbon emissions) will remain a central part of any response. Nordhaus presents estimates of the costs associated with this, from both economic and engineering perspectives, and explains the importance of discounting future costs. 

An overview of international policy history is followed by an explanation of the necessity for some kind of cost-benefit analysis and an illustration of the global balance between the costs of mitigation and the costs of climate change. Nordhaus explains how carbon prices work and why they are essential for a response flexible enough to cope with the complexities and uncertainties of economies and technologies. At a national level, he explains the advantages and disadvantages of carbon taxes and cap-and-trade systems. Internationally, he argues that a harmonised minimum carbon price will be easier to negotiate, and to maintain, than a cap-and-trade system such as Kyoto. 

The ``second best" alternatives to trading systems mostly involve regulation -- of, say, car fuel efficiency -- and are very much less efficient, sometimes even counter-productive. (The presence of ``energy-cost myopia", where consumers don't fully value savings from energy efficiency improvements, may provide some justification for regulation on other grounds.) Governments have an important role to play in supporting research and development of new technology in areas such as energy efficiency, but appropriate carbon pricing will also help here. 

A final section looks at ``climate politics", with chapters on criticism of climate science, public opinion, and obstacles to policies addressing climate change. This is perhaps the most US-centric part of The Climate Casino. 

Despite the subtitle, Nordhaus doesn't really attempt to convey the uncertainty of climate science. To give just one example, he includes a pair of figures showing the projected rise in sea levels through 2100 and 2500, for unconstrained and temperature-limited emission scenarios, which don't indicate any of the uncertainty from climate models or ice-sheet dynamics. Given the economic focus of The Climate Casino, this is a reasonable simplification -- and Nordhaus does consider, more abstractly, tipping points and ``unknown unknowns". The casino metaphor is also stretched a bit far -- at one point Nordhaus has to resort to having us imagine a roulette wheel with an unknown number of slots and unknown bad consequences on getting a zero! -- but it's not clear that there is a better alternative. 

Minor quibbles aside, The Climate Casino is a clearly presented introduction to an important topic. It will appeal to a broad audience, but should really be read by any politicians responsible for climate change policy.
