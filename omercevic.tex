\newpage\nuluj
\pagestyle{omercevic}

\logo

\nadpis{Introducing Good Money:\\Legal Tender Problem or Question\\of Structured Approach?}

\autor{Edo Omer�evi�\pozn{Center for the Advancement of Free Enterprise, (Centar za poslovnu afirmaciju), Dr. �ire Truhelke 10/A, 72000 Zenica, Bosnia and Herzegovina. E-mail: edo.omercevic@cpa-bih.org.}}
\linka{3ex}

\cast{Abstract}
The aim of this paper is to theoretically analyse the issues concerning the introduction of alternative moneys. A brief performance evaluation of the current monetary system, as well as two alternatives, namely gold and a private currency, i.e. the Swiss WIR, is followed by a historical look at the relationship between legal tender laws and Gresham's Law. The significance of legal tender laws in introducing alternative moneys is questioned and the focus is shifted to the political and ideological problems. It is pointed out that a~professional and structured approach needs to be adopted to successfully introduce monetary substitutes. The derived conclusion is that legal tender laws are not the major factor hindering the introduction of alternative moneys, but that a range of legal, economic and political factors require a dedicated and professional approach to enable the private sector to work on better exchange mechanisms.\\

\cast{Keywords}
Money, Legal Tender Law, Private Currency
\linka{4ex}

\podnadpis{Introduction}
Many voices have been raised against the legal tender law enjoyed by national currencies, accusing it of giving the beneficiaries a market advantage over alternatives, which, so it is argued, would outperform national moneys. The critics are sometimes very strident. Gnazzo (2009), for example, claims that the legal tender law has only one purpose: to ``\fntit{force the acceptance of the unacceptable}''; while Hornberger (2000) describes it as a~method tyrannical governments use to get their hands on the wealth and property of the people. Without evaluating the substance of the abovementioned claims, this paper will examine the effects of legal tender law on the introduction of alternative or complementary moneys.

Before looking into the matter of legal tender, a simple evaluation of the current system will be provided along with two potential alternatives, namely a return to commodity moneys or the introduction of complementary/alternative currencies and/or credit systems. The paper continues by providing the contemporary definition of legal tender, a brief historical comparison of its application and its relevance to Gresham's Law. What follows is an examination of the role legal tender law plays and what can be done to introduce good moneys, as perceived by the introducers.\\

\podnadpis{A brief performance assessment of the current monetary system}
As is widely agreed, money has three main functions upon which one can assess the performance of a chosen currency, namely that of unit of account, medium of exchange and store of value. Money as a unit of account is needed to set the prices of goods and services offered for exchange, expressing their relative economic values. A better unit of account would be one which is more reliable in quoting prices. This function is closely related to its function as a store of value, a feature which enables money to be used to transfer purchasing power from the present to the future. A stable unit of account is a requirement for a stable store of value. A convenient way for a layman to assess the performance of the current globally implemented interest based fiat monetary system (or in short ``{\bf debt money}'') in providing a stable unit of account and store of value, is to visit the website http://www.usinflationcalculator.com and check how much prices have changed in the United States. If one, for example, compares changes in price levels from 1980 to 2009, the inflation rate, meaning the increase in the price level (related to the function as a unit of account) or loss in purchasing power (related to the function as a~store of value) would be 162.5\%. Thus, an item that cost USD 1 in 1980 would cost USD 2.63 in 2009. A more shocking figure would be if one compares changes from 1913 to 2009. The result is an incredible 2085.2\% increase in the price level. Hence, an item that cost USD 1 in 1913 would be priced today at USD 21.85. The situation in the US is still tolerable compared to many countries in which inflation rates reach similar levels within just a~fraction of this time.

The performance of debt money as a medium of exchange might be evaluated based on its supporting role in the exchange process, as well as the cost of facilitating that process. Looking at the number of financial crises and periodic business cycles, one might at least doubt this performance. The cost of maintaining this doubtful system is, on the other hand, simply shocking. Without performing a deep analysis, one might just look at the fact that Germany has budgeted an incredible EUR 40,400,000,000 for interest payments (Rei�mann, 2009) for the year 2010 -- more than forty billion euro just to be paid in interest. It is estimated that this figure would allow Germany to, for example, build 3,333 km of new highways, or provide high-speed Internet to every household and catapult Germany into a high-tech future (Rei�mann, 2009). It is also worth noting that this figure does not include the cost incurred by the private sector.

\clearpage\newpage

\podnadpis{The acclaimed performers}
Some claim that commodity moneys, for example, gold, perform money functions better than debt money (H�lsmann, 2008; Buffett, 1948; Paterson, 1943). Others deny these claims of superiority (Gesell, 1918) or even go so far as to ridicule calls for a return to gold (Hasan, 2007). As much as the performance of gold as a medium of exchange might be debated, it seems that gold performed much better as a unit of account, and hence as a~saving medium. For example, as shown in figure 1, the oil price between 1973 and 2008 fluctuated when compared to the gold price as well as the dollar price, but the gold price fluctuated around a stable mean, while the dollar fluctuated around an increasing mean. The relatively higher price anchoring displayed by gold versus other goods of course also contributes to its role as a store of value. The performance of gold for storing wealth was so successful that Ibn Kahldun, for example, considered gold (and silver) to be God-given measures of value and to constitute the basis of profit, property, and treasure (Dawood, 2005).

Others claim that a return to commodity moneys is unnecessary. An alternative might simply be the implementation of complementary money and credit systems based on fiat mediums of exchange (Greco, 2001). The case of the Swiss WIR Bank could be used as an example, whereby the bank issues its own private currency, called the WIR, parallel to the official Swiss franc. WIR money circulates only among members (loans are much cheaper than those paid out in Swiss francs) and is used to strengthen trading ties between participants mostly limited to small- and medium-sized enterprises. Stodder (2007) even found evidence that the WIR money supply is countercyclical (while the national currency is cyclical) and contributes to maintaining price stability. However, given that the WIR supply constitutes only about 1\% of the national M1 money supply, the positive effects might not be shown to their full extent. 


\popis{Figure 1: Dollar vs. gold price of crude oil}
\obr{omercevic-fig01.jpg}{\textwidth}
\zdroj{Source: Bloomberg \& www.inflationdata.com}

Looking at the growth pattern between the general M2 and the WIR credit as shown in figure 2, it can be seen that the M2 imitates a rather linear growth pattern, which would require linear GDP growth if inflation is to be avoided, while the WIR credit is somewhat normal, having started on an exponential growth pattern, simulating a natural growth limitation. Hence, the WIR growth path seems to be more adapted to human nature than the official national currency. And yet, in Switzerland and elsewhere, national currencies prevail versus commodity and alternative fiat currencies. The response to those seeking alternative moneys could therefore be that, if the alternatives offered really perform and meet the requirements of exchanges economies (or societies) better, market participants would make the shift away from inferior national debt moneys. So why do they not?

\popis{Figure 2: Swiss M2 vs. WIR outstanding credit and growth patterns}
\obr{omercevic-fig02.jpg}{\textwidth}
\zdroj{Source: Swiss National Bank, Stodder (2007)}

\podnadpis{Legal tender law and Gresham's Law}
The widely agreed upon contemporary definition of legal tender is that it is ``\fntit{\dots money which a creditor is legally obliged to accept in settlement of a debt (Black, 1997, p.\;266).}'' Hence, the contemporary legal tender law is actually intended to regulate the settlement of debts. This was not always the case. In the era of metallic moneys, legal tender law has been used to regulate the acceptance of coins which had variations in their metal content, resulting in different intrinsic values, while enjoying the same purchasing power in the market, power which was based on the face value as guarantied and enforced by the privilege of the legal tender status. As such, the clipping of coins was a common way for governments and sovereigns to increase their finances at the cost of the general population, as was the modification of coin alloys or even the issuance of coins with the same face value but produced fully from a much cheaper metal (e.g. the introduction of copper coins as a third metal in gold and silver denominated monetary systems). For example, medieval English sovereigns would give themselves the privilege of being the sole authorized party for coining money and sometimes go as far as to stipulate penalties for refusing to accept face value irrespective of intrinsic content. As mentioned by Allen (1999), referring to medieval English sovereigns, the:

\clearpage\newpage

\text{``\fntit{\dots orders from the crown went so far as to require the acceptance of pennies that had been halved and demanded that anyone refusing to accept half pennies should be seized for contempt of the king's majesty, imprisoned, and exposed to public ridicule in a pillory (p.\;183).}''}

The effect of such actions would, of course, result in a preference for coins of lower quality to be used for exchange purposes. Should lower quality coins reach a level of supply high enough to cover all the transactions taking place in the market, coins with a higher intrinsic value would cease to circulate and it might be observed that lower quality coins drove out those of better quality. Such a development is described as Gresham's Law, which states that ``bad money drives out good''. 

In bimetallistic monetary systems, the legal tender law was also used to regulate exchange rates between the different types of coins, which was usually  different from the market exchange rate and hence resulted in arbitrage opportunities; sometimes the effects would be so strong that the overvalued currency would drive the undervalued one out of circulation, causing an evolution into a monometallic monetary system. Legal tender law has also been applied to fiat moneys and credit moneys. In fact, it has been pointed out that fiat moneys in particular enjoy purchasing power only by means of legal tender privileges (H�lsmann, 2004). Hence, legal tender law has been employed for all types of moneys and in various forms.

Based on its historical track record, it might be argued that market participants are not shifting to the `better' moneys because of the existing legal tender privileges national currencies enjoy, given that legal tender law historically led to the Gresham effect. However, as has been pointed out by many, Gresham's Law comes into existence only in those cases where the attached legal value does not correspond to the market determined purchasing power (Strand, 2001; Selgin, 2003; Mundell, 1998). The classic example would be two gold coins circulating in the market, both with the same face value but with different intrinsic values, or gold content. Imagine two gold coins with a face value of 1 dinar, one having a gold content of 4.25 grams and another which only contains 3 grams. Assuming both coins can be utilized at a face value of 1 dinar, and that you are a very economical person, you will tend to keep the coin with the higher gold content and use the other for the payment of a 1 dinar bill. You might thus accumulate many 4.25 gram coins, melt them down and make more coins with a content of 3 grams, a process of an easy enrichment, provided the coin can still be used at its face value of 1 dinar. As a result, in aggregate, if enough lower quality coins are available, the higher quality coins would cease to circulate in the economy. Such misuse has indeed taken place, as mentioned at the beginning of the paper. However, if the value of the coins is left to be determined by the market based on the gold content, the lower quality coins would simply be priced below their face value, according to the extent of their inferiority. Gresham's Law would not materialize, as the situation would be similar to that of coins with different face values, like 1\;cent and 10\;cent coins, in which the 1 cent coin does not drive the 10 cent coin out of circulation. Hence, by having two circulating moneys in an economy, the bad one would be driving the good one out of circulation only in those cases where the face value differs from the intrinsic value, or if the exchange rate is set at a different rate than the market determined rate. 

But how is the logic for previous monetary arrangements to be applied to the contemporary scenario? As mentioned above, today's widely applied legal tender takes the form of the granting of a debt settlement function to national currencies and not regulating their purchasing power. Hence, is the legal tender law really stopping market participants from shifting to better moneys (or what they perceive as better)? 


\podnadpis{Legal tender law and Gresham's Law}
As mentioned earlier, today's legal tender law gives absolute power to national currencies to act as mediums for all debt settlements. However, the situation has changed in one major aspect. Previously, a connection existed between money and commodities (either they would be converted into, or they would have a fixed convertibility into, commodities). Thus, for example, a dollar would have a convertibility rate into gold (or silver). Hence a legal tender ensuring purchasing power equivalent to the face value of money would mean that the issuer of the notes was able to purchase with the notes according to the purchasing power of the reference commodity rather than the money's intrinsic value. In such a scenario, Gresham's law would come into play and, if the supply was sufficient, the `worthless paper' would drive out of circulation any gold or silver coins. In a similar manner, if two commodities circulated as money under a fixed exchange rate, at which the nominal does not correspond to the real exchange ratio, the overvalued currency would drive out the undervalued one.

However, today's paper money has no commodity reference: one United States dollar is worth one dollar, whatever that might mean. Thus, is there a difference between the face and intrinsic value of the monetary unit? Such a difference would exist in the event that the purchasing power is expressed in terms of the cotton fibre paper from which the dollar bills are made, and if the expressed face value was different from the intrinsic value of the money. But the face value of the dollar is not expressed in terms of cotton paper, but in terms of dollars. The cotton seller, and any other seller, is free to determine how many dollars to demand or even to reject the dollars altogether.\pozn{What must be mentioned here is that country and time differences exist. While today, for example, parties in most countries are allowed to mutually decide on what payment modes and forms to use, while payment with legal tender only ensures that a debtor cannot successfully be sued for non-payment if he pays into court in legal tender (TRM, 2010), acceptance might be obligatory at some places at any stage of a transaction, even before the occurrence of debt.} As a result, it should actually be possible to introduce alternative moneys. And the current real world scenario confirms that possibility. The previously mentioned WIR Bank has been successful in issuing its officially recognized alternative currency, the WIR; the internationally successful barter company Bartercard issues its T\$ (trade dollars) and both Europe and the North American continent have witnessed the existence of hundreds of private moneys as well as many Local Exchange Trading Systems (LETS)\pozn{Germany alone has about 16 alternative currencies in circulation, and another 60 known initiatives as of 2006 (R�sl, 2006). LETS are even too many to be counted and exist in almost any mid- and big-size city.}. In the United States of America, people were able to use Liberty Dollars, alternative silver and gold based monetary units. Most alternative fiat or credit moneys do not have their own value, but refer to the national debt money as a reference unit, and yet some, like the Liberty Dollar, have their own value and hence an exchange rate with the national debt money. Having a private credit or fiat currency can save users a lot in interest and, with a decoupled and independent valuation, private currencies can to a great extent protect their bearers from the inflationary pressures exercised via national debt moneys. In short, legal tender law, as applied today in most countries, does not represent an unbreakable barrier to alternative money systems.

\podnadpis{It is a political and ideological problem}
As mentioned above, the legal tender law is not a barrier to new and alternative currencies, money, and exchange systems to be introduced into the market. However, it might be said that the success of all alternatives has been rather negligible. Is this simply because of the intrinsic weakness or fundamental inferiority of the offered alternatives? Is it simply a fact that national debt moneys, despite their bad performance and expensive maintenance, still outperform any alternatives?

The answer is not simple. To accuse the alternative models of inferiority would be justified if they failed all by themselves. However, the facts actually indicate the opposite. In many success stories of alternative systems, it is the interference of the debt money managers (central banks, governments or their mints) which influence or terminate the existence of successful alternative models. Hence, on the international level, it is very obvious that it is difficult for gold to celebrate a return if the yellow metal has been declared the only item countries cannot use for defining the value of their currencies (Weiss, Martin A. \& Sanford, Jonathan E., 2007). Even on the national level the movement is facing difficulties. Thus, the US Mint claimed that the usage of Liberty Dollar's gold and silver coins constitutes a Federal crime (Liberty Dollar v. US Mint, 2007). But it is not only metallic moneys which face difficulties -- so do fiat and credit moneys. An experiment took place in 1932--33 in the Austrian city of W�rgl to introduce a local currency with the aim of ending local unemployment and strengthening the local economy but was ended by force by the Austrian government, despite its recorded success and strong local interest and support (Unger, 2010). The WIR money issued by the WIR Bank is under the regulation of the Swiss National Bank and subject to their requirements, and Germany's central bank only just tolerates the various existing currencies, but monitors the ``social cost'' caused by them (R�sl, 2006). It might be assumed that, should the mentioned `social cost' (which in reality could be understood to be a figurative synonym for market penetration) become too significant, the State might intervene, as was the case in W�rgl in Austria. 

In reality, the whole matter might be a struggle between ideologies, between one which centres economic control in the hands of governments and the banking sectors and the other which stands for decentralized control over economic growth patterns and structure as well as decentralized and localized credit creation power. The implications of this struggle are immense and will determine the future of our lives; the way our economic structure will be determined, either by aggregate decision-making process by each market participant, or by a group of people setting the economic targets for the rest of the society; the way our savings are mobilized, forcefully via inflation or democratically via incentives; the way credit is issued, at a price set by central banks, or at a price determined by the actual creditors; and more important implications. The significance of this struggle has been put into words by an economist from the German School of Money and Banking, Ulrich von Beckerath (1882--1969):

\text{``\fntit{Extension of exchange transactions without State money is in reality the beginning of a~new system of settling accounts, \fntbf{indeed the beginning of a new economic order} (Greco, 2004, p.\;3, bold added).}''}\vspace*{2mm}

\podnadpis{How to introduce alternatives?}
As discussed above, the legal tender law `only' gives national currencies the power to act as settlement items for outstanding debts. However, such a function is very powerful for influencing the decision-making process of market participants on the exchange medium they are going to use and how they will express their contractual obligations. The fact is that national currencies can be legally used to settle past, present, and future liabilities, which makes them an excellent tool to quote and of course settle obligations. Besides settling outstanding debts, governments create a market for their currencies by making taxes payable only in national moneys and it might be assumed that no companies with links to government (mostly huge market participants on which a lot of businesses depend) would accept dealing in a currency which is competing with the national ones. Hence, the legal tender law is just one out of a set of means to ensure the monopoly of national currencies. And as mentioned above, even when an alternative currency succeeds in attracting large enough participation, governments might simply take legal action against it. That, however, would be a last resort, and one which does not have to happen often, given all the privileges national currencies enjoy in the market, all of which make the emergence of alternatives a very challenging task. 

Given the strong competition faced in the form of national currencies, Greco (2009) points out four factors which influence the success (or failure) of an issuing entity (the ``\fntbf{alternative provider}''):
\zacseznam
\item The architecture of the alternative provider or the alternative itself
\item The management of the alternative provider
\item The implementation strategies
\item The context into which the alternative is introduced
\konseznam

It is very important not to burden the alternative provider with objectives which might endanger the main objective -- supporting the exchange process. Trying to solve all problems perceived by alternative providers might jeopardize the success of their undertaking. The issue as to who is authorized to supply alternative money must be clearly and transparently defined, as well as the basis and the limits of issuance.

It is also very important to plan how the alternative is to be promoted among potential users. A critical mass of participants is decisive for the sustainability of the alternative. Should a critical mass not be achieved within a reasonable time period, the alternative might lose its attractiveness and even fail to secure its continued existence.

The timing of the introduction of the alternative is also very important and must coincide with the appropriate situational context. Introducing an alternative during a period of easy money and an economic boom might be more difficult than during a period of financial scarcity and economic depression. Hence, timing is definitely a factor which might influence the success of an alternative.

In addition to Greco's four factors, one more should be added, namely the legal aspect. It is very important to make sure the law is satisfied when launching alternative moneys. Laws differ from country to country, so what is left for those looking into the possibility of establishing a private monetary alternative is to structure the alternative provider in compliance with the legal standards of the host country and try to look for spaces in the law which enable such private initiatives.

Freire (2009) provides an excellent overview of the legal challenges to establishing alternative monetary mediums (termed as ``\fntbf{social currencies}''), but only in the case of Brazil. A taste is given of what considerations need to be observed when deciding on the organizational structure of the alternative provider:

\text{``\fntit{Under Brazilian legislation, in principle, not-for-profit non-government organizations (NGOs), Organizations of Civil Society in the Public Interest (OSCIPs), and Municipal Funds do not require authorization from the Government or from the Central Bank to carry out projects entailing social currencies. Municipal Funds, however, are subject to limitations under the Fiscal Responsibility Law (Enabling Law (LC) 101, of 2001), particularly with respect to the assuming of obligations within the scope of social currency systems.\\
If social currencies are linked with microfinance programs, both NGOs and Municipal Funds are subject to limitations under the Usury Law (Dec.\;22.626/33), whereas OSCIPs are not (Executive Order (MP) 2.172-32/01). The situation is different when social currencies are established for profit-making ventures. In this case, it would be generally necessary to obtain authorization from the Government or from the Central Bank. For example, a benefits program involving the issuing of bonds or vouchers that serve as a~social currency might require an authorization by the Ministry of Finance (Law 5.678/71), whereas a~Credit Society for Micro-businesses or a Credit Cooperative that wishes to issue a social currency must seek authorization from the Central Bank (p.\;87, 88).}''}

Hence, it is highly advisable to get legal advice before deciding on the legal structure of the alternative provider in order to avoid future legal disputes. Once it has been legally approved, it will surely be more difficult to challenge the legality of the alternative at a~later date.\pozn{Unfortunately, in some countries the laws against alternative currencies are very rigid. Thus, for example, in Canada's 1980 statute it is stated that ``Every bank or other person who issues or reissues, makes, draws or endorses any bill, bond, note, check or other instrument, intended to circulate as money, or to be used as a~substitute for money, is guilty of an offense against this Act (Rozeff, 2009).''}

\podnadpis{Conclusion}
To conclude this brief paper on introducing alternatives which are perceived by users as good or better moneys, in the current monetary framework legal tender laws in most of the world might not be a legal barrier that cannot be breached to introduce alternative moneys under private initiative. Even though legal tender laws have been historically abused to impose mediums of exchange on market participants, in the contemporary scenario it does not represent the main hindrance to introducing alternatives. Unfortunately, a much broader and wider political and economic scenario has been created, with the impact that the introduction of alternatives might be legal, but economically difficult to realize. Hence, even in a situation in which the legal framework allows for the introduction of private exchange mechanisms, what is equally necessary is a serious and professional approach which requires human and capital commitment. However, besides the efforts made by alternative providers, it must be clear that, in the end, the limits of growth, development and finally the success of alternatives are subject to the willingness of the respective governments to allow such developments to take place and to accept the decentralization of money and credit control. 

But no change happens overnight, and especially not a change in power. As much as the struggle will be difficult, a transformation to a new economic order, one in which economic and credit power is decentralized and based on diversified views on credit and exchange processes can take place as long as persistence towards change continues.\\[4mm]

\cast{References}
Allen, L. (1999). \fntit{Encyclopedia of Money}. Santa Barbara, USA: ABC-CLIO Inc.

Black, J. (1997). \fntit{Oxford Dictionary of Economics}. 2nd Edition. New York, USA: Oxford University Press Inc.

Buffett, H. (1948). \fntit{Human Freedom Rests on Gold Redeemable Money}. Reprinted from The Commercial and Financial Chronicle 5/6/48. Retrieved December 22, 2009, from http://fame.org/PDF/buffet3.pdf.

Dawood, N.\;J., Ed. \& Abridg. (2005). \fntit{Ibn Khaldun, The Muqaddinah: An Introduction to History}. Translated by F.\;Rosenthal. New Jersey, USA: Princeton.

Freire, M.\;V. (2009). Social Economy and Central Banks: Legal and Regulatory Issues on Social Currencies (Social Money) as a Public Policy Instrument consistent with Monetary Policy. \fntit{International Journal of Community Currency Research}, 13, 76--94.

Gesell, S. (1918). \fntit{The Natural Economic Order}. Translated by Philip P.M.A [e-book]. Retrieved April 30, 2008, from
http://www.scribd.com/doc/12347428/GesellSilvioTheNa\-turalEconomicOrder\#document\_metadata.

Gnazzo, D.\;V. (2009). \fntit{The Constitution: On Legal Tender \& Lawful Money}. Financial Sense [Online] (Updated 10th November, 2009). Retrived November 19, 2009, http://www.financialsense.com/fsu/editorials/gnazzo/2009/1110.html.

Greco, T.\;H.\;Jr. (2001). \fntit{Money: Understanding and Creating Alternatives to Legal Tender}. Canada: Chelsea Green Publishing Company.

\clearpage\newpage

Greco, T.\;H.\;Jr. (2004). \fntit{The Need for Monetary Reform and the Status of the Movement}. Retrieved
April 14, 2009, from http://www.reinventingmoney.com/documents/Mone\-taryMo\-vementStatus.pdf.

Greco, T.\;H.\;Jr. (2009). \fntit{The End Of Money And The Future Of Civilization}. Vermont, USA: Chelsea Green Publishing Company.

%\clearpage\newpage

Hasan, Z. (2007). \fntit{Ensuring exchange rate stability: Is return to gold (Dinar) possible?} Munich Personal RePEc Archive [Online] (Updated 8th April, 2008). Retrieved January 31, 2009, from http://mpra.ub.uni-muenchen.de/8134.

Hornberger, J.\;G. (2000). \fntit{Legal Tender and the Civil War}. The Future of Freedom Foundation [Online]. Retrieved
November 23, 2009, from http://www.fff.org/freedom/1100a.asp.

H�lsmann, J.\;G. (2004). Legal Tender Laws and Fractional-Reserve Banking. \fntit{Journal of Libertarian Studies}, 18(3), 33--55.

H�lsmann, J.\;G. (2008). \fntit{Mises on Monetary Reform: the Private Alternative}. The Quarterly Journal of Austrian Economics [Online] (Updated 13th November, 2008). Retrieved March 15, 2009, from http://mises.org/journals/qjae/pdf/qjae11\_3\_3.pdf.

Liberty Dollar v. US Mint (2007). \fntit{Bernard Von Nothaus and Liberty Dollar v. Henry M.\;Paulson, Jr., Alberto R. Gonzales and Edmond C. Moy} [2007] 3:07-CV-038 RLY-WGH. Retrieved February 19, 2010, from http://www.libertydollar.org/legal/pdf/complaint.pdf.

Mundell, R. (1998). \fntit{Uses and Abuses of Gresham's Law in the History of Money}. Columbia University [Online]. 
Retrieved November 25, 2009, from http://www.columbia.edu/\%7E\-ram15/grash.html.

Paterson, I. (1943). \fntit{Why Real Money Is Indispensable}. From: The God in the Machine, Chapter XVIII. Retrieved December 22, 2009, from http://fame.org/PDF/Indispen\-sable.PDF.

Rei�mann, O. (2009). \fntit{Deutschland in der Zinsfalle: So sch�n lebt sich's schuldenfrei}. Spiegel Online [Online] (Updated 16th December, 2009). Retrieved December 17, 2009, from http://www.spiegel.de/wirtschaft/soziales/0,1518,667527,00.html.

R�sl, G. (2006). Regional currencies in Germany -- local competition for the Euro? \fntit{Deutsche Bundesbank, Discussion Paper Series 1, Economic Studies}, 43.

Rozeff, M.\;S. (2009). \fntit{Why Do Central Banks Exist?} Social Science Research Net\-work. Retrieved
April 8, 2010, from http://ssrn.com/abstract=1500263.

Selgin, G. (2003). \fntit{Gresham's Law}. EH.Net Encyclopedia [Online], edited by Robert Whaples.
Retrieved November 25, 2009, from http://eh.net/encyclopedia/article/selgin.gre\-sham.law.

Stodder, J. (2007). \fntit{Residual Barter Networks and Macro-Economic Stability: Switzerland's Wirtschaftsring}.
Retrieved August 15, 2008, from http://www.ewp.rpi.edu/hartford/\~{}stod\-der/MacroPrinc/Stodder\_WIR.pdf.

Strand, J.\;R. (2001). ``Gresham's Law'', in R.\;J. Barry Jones (Ed.), \fntit{Routledge Encyclopedia of International Political Economy}. London, UK: Routledge, p.\;638.

TRM (2010). \fntit{Legal Tender Guidelines}. The Royal Mint (TRM). Retrieved February\;4, 2010, from
http://www.royalmint.com/corporate/policies/legal\_tender\_guidelines.aspx.

\clearpage\newpage

Unger, B. (2010). \fntit{Besser Wirtschaften: Regionale Utopie}. Zeit [Online] (Updated 13th~January, 2010).
Retrieved January 20, 2010, from http://www.zeit.de/online/2007/37/besser-wirtschaften-brigitte-unger.

Weiss, M.\;A. \& Sanford, Jonathan E. (2007). \fntit{International Monetary Fund (IMF): Financial Reform and the Possible Sale of IMF Gold}. The Library of Congress. Retrieved February 19, 2010, from http://fpc.state.gov/documents/organization/93480.pdf.

